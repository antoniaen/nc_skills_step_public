
While numerous studies establish positive effects of education on cognitive skills \parencite[e.g.][]{schneeweis_does_2014, dahmann_how_2017, cappellari_effects_2023}, much less is known about the effect of education on personality traits, preferences, and attitudes. These personal attributes are often referred to as non-cognitive skills because they are rewarded in the labor market \parencite{bowles_incentive-enhancing_2001} similar to cognitive skills. With technological advancements potentially substituting human cognitive abilities and the increasing complexity of organizing work, the importance of non-cognitive skills is expected to grow, leading to higher returns for these skills \parencite{deming_growing_2017, edin_rising_2022}. Non-cognitive skills, along with cognitive skills, significantly influence socioeconomic outcomes such as employment, wages, and health \parencite{heckman_effects_2006, heckman_understanding_2013}, including in low- and middle-income countries \parencite{sharma_does_2018, gertler_labor_2014, glewwe_cognitive_2017}. Despite being similarly predictive of socioeconomic outcomes as cognitive skills, and increasingly important in the labor market, it remains an open question whether compulsory education affects non-cognitive skills. We cannot simply infer findings from studies on cognitive skills to non-cognitive skills as these attributes may respond differently to educational inputs. Previous research, such as the studies by \citet{heckman_understanding_2013} and \citet{alan_ever_2019}, indicates that non-cognitive skills are affected by early childhood programs and curriculum interventions, therefore, one could expect that compulsory education has an impact as well. We aim to investigate whether this is indeed the case and how various non-cognitive skills are affected. Analyzing this relationship is crucial, as it would provide insights into whether and how education systems can nurture traits that are increasingly valuable in the economy.


In this paper, we analyze the long-term effects of compulsory schooling reforms on non-cognitive skills.\footnote{
	In Table \ref{tab:res_lit_test} in the Appendix section with findings for additional outcomes, section \ref{sec:app_add_outcomes}, we also report on the effect of compulsory schooling reforms on cognitive skills.
} We examine five educational reforms that mandated compulsory primary or lower secondary education in four low- and middle-income countries: Bolivia, Colombia, Ghana, and Vietnam. Our analysis relies on cross-sectional World Bank data from 2012/13, which is representative of individuals aged 15-64, living in urban regions of these countries. The dataset includes measures of the Big Five personality traits, risk and time preferences, grit, hostile attribution bias, and decision-making patterns. Moreover, scores from a literacy test are included. For causal identification of effects, the cohort trends in these skills must be well approximated and no other event should differently affect treatment and control groups. We assess the returns to non-cognitive and cognitive skills by estimating augmented Mincer Wage regressions.

We find that the reforms expanding compulsory education affected non-cognitive skills. Specifically, the reforms decreased emotional stability, grit, hostile attribution bias, willingness to take risks, and patience, while increasing openness to experience and alternative solution or consequential thinking. In our sample, increased openness to experience and reduced hostile attribution bias are positively related to wages, whereas a reduced willingness to take risks is negatively associated with wages. The other skills affected by the reforms do not show a significant relationship with wages. Furthermore, we find no evidence that individuals affected by the reforms differ in their probability or type of employment compared to those not affected. Considering the effects on years of education and cognitive skills, our correlational results from the augmented Mincer wage regression suggest that educational expansions, on average, increase wages by 9\%, with 40\% of this predicted increase stemming from changes in non-cognitive skills. We discuss potential mechanisms---such as educational quality, perceived importance of education, and ability mixing---to better understand how compulsory schooling reforms might influence the development of non-cognitive skills.

To the best of our knowledge, this is the first study to examine the causal effect of reforms raising years of compulsory education on personality traits, grit, hostile attribution bias, and decision-making patterns. Our study complements existing research that has documented the causal effects of curriculum interventions on specific character traits. For example, \citet{alan_ever_2019} and \citet{santos_can_2022} show that school interventions aimed at enhancing children's grit can be successful. Other related studies investigate the effects of different school types, such as a selective college or vocational versus general secondary education \parencite{dasgupta_effects_2022, ollikainen_effect_2024} on personality, or the effects of special educational interventions, such as a German school reform or an equalization policy in Korea that resulted in ability mixing within classes \parencite{dahmann_impact_2014, dahmann_cross-fertilizing_2018, ahn_long-term_2021}. For instance, \citet{dasgupta_effects_2022} find that attending a selective college decreases men's extraversion and conscientiousness, while \citet{dahmann_cross-fertilizing_2018} find that the introduction of G8 in Germany (a reduction in the years of schooling while keeping the same curriculum, which led to an increased intensity of schooling) decreased individuals' emotional stability.


Furthermore, this study contributes to the literature on the effect of education on economic preferences, which has yielded mixed findings. \citet{jung_does_2015} finds that a school reform increasing compulsory education in Great Britain decreased individuals' willingness to take risk, whereas \citet{tawiah_does_2022} reports that a similar reform in West Germany increased the willingness to take risk. Our study is the first to provide evidence on the effect of education on willingness to take risks in low- and middle-income countries, and our finding of a negative effect aligns with \citeauthor{jung_does_2015}'s \parencite*{jung_does_2015} results for Great-Britain.

Regarding time preferences, three studies provide relevant empirical findings. \citet{dohmen_effect_2022} show that compulsory schooling reforms positively affect patience across 48 countries, with this overall positive effect driven by developed countries, while there is a null effect for developing countries. Similarly, \citet{jung_does_2021} find a positive effect on patience, identifying this effect through increased years of education induced by reduced schooling costs in Indonesia. In contrast, \citet{tawiah_does_2022} finds a negative effect on patience.

Finally, we contribute to the literature on labor market returns to non-cognitive and cognitive skills.\footnote{
	We also show results for the effect on cognitive skills, in particular, on literacy test scores (Table \ref{tab:res_lit_test}). Thereby, we contribute to the literature on the causal effect of compulsory schooling reforms on cognitive skills in non-OECD countries. While numerous studies from the fields of psychology and economics investigate the effect of schooling on cognitive skills, they usually provide evidence from OECD countries \parencite[see, for instance,][]{schneeweis_does_2014, cappellari_effects_2023}.
} For instance, \citet{heckman_understanding_2013} study how the Perry Preschool program for disadvantaged African Americans boosted individuals' labor market outcomes, showing short-term effects on IQ and long-term effects on externalizing behavior and academic motivation. \citet{heineck_returns_2010} investigate the wage-returns of the Big-Five personality traits, locus of control, reciprocity, and fluid intelligence using data from Germany. \citet{hanushek_coping_2017} study cross-country differences in returns to cognitive (numeracy) skills using data from the PIAAC survey. \citet{piopiunik_skills_2020} use fictitious resumes to study which skills improve employability. \citet{heckman_effects_2006} analyze the effects of both cognitive and non-cognitive skills, modeling both types of skills as latent traits, and study the effect on wages, work experience, and occupational choice. They find that both types of skills significantly impact labor market outcomes.


The paper is structured as follows. Section \ref{sec:reforms} provides a detailed overview about the compulsory schooling reforms. Section \ref{sec:data} introduces the data used in our analysis. Section \ref{sec:emp_strategy} outlines the research design, the empirical approach, and the effects we expect to observe based on the literature. Section \ref{sec:results} presents the results, provides robustness checks of the results, and evaluates the findings based on estimated wage returns to skills. Section \ref{sec:discussion} examines the potential mechanisms through which expanding compulsory education could impact non-cognitive skills and discusses the limitations of the analysis. Section \ref{sec:conclusion} concludes and considers the implications of our findings for the malleability of non-cognitive skills.
