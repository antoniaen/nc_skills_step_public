Our results show that reforms increasing compulsory education significantly affect personality traits, economic preferences, decision-making patterns, grit, and hostile attribution bias of treated individuals. The positive effect on openness to experience and the negative effect on hostile attribution bias align with our expectations based on correlations reported in the literature (see Section \ref{subsec:exp_effects}).

In contrast, the negative effect on emotional stability contrasts with our expectations. One possible explanation is that education, by broadening individuals' knowledge of the world and their own limitations, may increase their tendency to worry, thereby reducing emotional stability. The negative impact on patience is also rather surprising given prior research, though evidence on this relationship remains limited for low- and middle-income countries. For instance, while \citet{dohmen_effect_2022} overall find a positive effect of compulsory schooling reforms on education, they report no significant impact on patience in low- and middle-income countries. The relatively low life expectancy in low- and middle-income countries might help explain this result. Research indicates that life expectancy increases both patience \parencite{becker_endogenous_1997, falk_longevity_2019} and educational attainment \parencite{ben-porath_production_1970, cervellati_life_2013}. In settings with shorter life expectancy and potentially lower educational quality, being compelled to stay in school longer---rather than choosing one's optimal level of education---may lead some individuals to perceive additional years as lost time, ultimately reducing their patience. This interpretation suggests that the effect of compulsory education on patience may be context-dependent.
The negative effect on grit is also unexpected. However, given the reduction in patience, a decline in grit---defined as passion and perseverance for long term goals---seems not surprising. Finally, our analyses show that expanding compulsory education decreases willingness to take risk, consistent with the findings by \citet{jung_does_2015} for Great-Britain.

We acknowledge that compulsory schooling reforms likely have heterogeneous effects on students. In particular, we expect these reforms to primarily, though not exclusively, affect individuals with lower levels of education and potentially disadvantaged socio-economic backgrounds. Therefore, our estimates may not be fully generalizable to the entire population but rather to individuals whose characteristics are similar to those increasing their years of education due to the reform. From a policy perspective, however, this subgroup is highly relevant. Beyond the increase in years of education, other potential mechanisms may contribute to our findings. We discuss these mechanisms below, followed by a consideration of data limitations.


\paragraph{Potential mechanisms} \label{sec:mechanisms} \mbox{}\\
Non-cognitive skills could be affected by reforms extending mandatory education through mechanisms beyond the direct increase in years of education. As mentioned in section \ref{subsec:conc_frame}, we explore these other potential mechanisms here.

First, the reform-induced increase in student numbers might strain educational resources, leading to overcrowded classrooms or teacher shortages, and thereby causing a decline in educational quality. Such a decline could adversely affect the non-cognitive skills of students who would have attended school even in the absence of the reforms.\footnote{
	To assess whether  our findings are driven by a decline in educational quality for more educated individuals, we split the sample based on years of education. The sample of less educated individuals is smaller. For both groups, we find no significant effect on agreeableness, extraversion, conscientiousness, or grit. However, emotional stability declines significantly in both groups, with a stronger negative effect for less educated individuals. Openness to experience increases significantly in both groups. Patience decreases similarly across groups, but the effect is only significant for more educated individuals. Hostile attribution bias decreases significantly for more educated individuals, while willingness to take risks declines significantly for less educated individuals. Finally, decision-making patterns improve significantly for more educated individuals.
}
Therefore, it is crucial to determine whether the government anticipated the challenges posed by the reform and responded by increasing the number of school buildings and teachers accordingly. Our data does not include direct measures of school quality, so we rely on information from relevant research papers and reports on the reforms. For four out of the five reforms studied in this article, we found pertinent information. In Ghana, the ``Accelerated Development Plans'' (ADP) for education, launched in 1951, ten years before the compulsory schooling reform \parencite{kadingdi_policy_2006}, lead to a rapid and sustained growth in schools and the number of teachers. Additionally, new teacher training colleges opened in 1961. However, \citet{kadingdi_policy_2006} notes that ``even though school enrollments increased following the 1961 Education Act, the quality of teaching and learning appears to have remained the same.'' In 1987, Ghanaian government received World Bank credits to finance the expansion of mandatory education, including new infrastructure \parencite{kadingdi_policy_2006}. In Colombia, an ambitious school construction initiative accompanied the 1991 education reform \parencite{urbina_mass_2022}, while the 1991 reform in Vietnam saw increased investments aimed at building new schools, training additional teachers, and providing financial support to disadvantaged students \parencite{cornelissen_multigenerational_2022, nguyen_trends_2004}. These examples suggest that maintaining educational quality was a key priority for the governments overseeing these reforms.


Second, increasing the years of mandatory education might signal the importance of education to parents or students. A student aware of the government's emphasis on education might feel more motivated to study, potentially enhancing educational outcomes. On the other hand, if the student distrusts the government, they might resist this directive, making the impact of this channel on student learning ambiguous. Parents who perceive education as more important after the reform might invest more in their children's education and possibly encourage them to pursue higher education. Our data set includes a variable measuring parental investment in education, specifically whether parents actively monitored their children's exam results or grades. As shown in Table \ref{tab:descriptives}, parental investment is significantly higher among treated individuals. However, after controlling for birth year and reform fixed-effects, this difference is no longer statistically significant (see Table \ref{tab:balanced}).

Third, mandatory education leads to a more diverse mix of students in school. For instance, children from a low-socioeconomic background, who might not attend school in absence of mandatory education, now do so. As a result, students affected by the reform are exposed to peers with different backgrounds and abilities. \citet{ahn_long-term_2021} find that ability mixing in education reduces individuals' agreeableness and conscientiousness. However, we observe no effect on these traits in our data. This could indicate that ability mixing was not particularly prevalent as a result of the reforms, or that any potential positive impact of education on conscientiousness and agreeableness was offset by the effects of ability mixing.


\paragraph{Limitations} \label{sec:limitations} \mbox{}\\
Our study has several limitations, primarily stemming from data-related challenges.

First, measuring non-cognitive skills is inherently challenging and prone to measurement error, especially when using a short version of a personality inventory with only three items per trait. While such short inventories are valuable in time-limited surveys, having fewer items increases measurement error \parencite{dohmen_accounting_2024}. Similarly, in the STEP data, skills are measured with just two to four items, which likely increases our standard errors due to measurement error in the dependent variable. Additionally, measuring skills in low- and middle-income countries poses extra challenges \parencite{bauer_using_2020, laajaj_challenges_2019}. \citet{laajaj_challenges_2019} found that a clear Big Five factor structure is often lacking in data from non-WEIRD populations, suggesting that survey questions may capture multiple traits instead of specific ones. Despite these challenges, our findings remain robust when replacing specific survey items as shown in column (3) of Table \ref{tab:robust_ncogn}.\footnote{
	\citet{laajaj_challenges_2019} propose alternative measures for the Big Five personality traits based on the same data as we are using. Using a psychometric approach, the authors show that---except for emotional stability---the survey questions do not always measure the intended trait. Instead, they may capture attributes of a different trait. To address this, we leverage \citeauthor{laajaj_challenges_2019}'s \parencite*{laajaj_challenges_2019} findings in Table 1 to replace survey items that capture attributes of other traits with items intended to measure other traits but that also capture attributes of the given trait. The results of this adjustment are presented in column (3) of Table \ref{tab:robust_ncogn}. Our main findings remain robust to replacing these survey items. Additionally, we find a positive effect on agreeableness of increasing years of compulsory education.
} \citet{laajaj_challenges_2019} also note that the lack of a clear Big Five factor structure in survey data from non-WEIRD populations is not due to education levels, but rather the influence of social desirability in face-to-face interviews.


Another limitation is that we observe individuals' skills only once, typically when they are in their 30s. While our empirical design enables us to attribute the estimated differences between treatment and control groups to the increase in compulsory education years, it does not allow us to disentangle the channels affecting skills during education from those influencing skills during labor market participation. For instance, an individual who obtained more education due to the reform might also work in a different occupation. Likewise, the larger number of educated young individuals could alter labor market competition. As a result, rather than pinpointing the specific channels through which mandatory education affects non-cognitive skills, we estimate the equilibrium effect of the reform.

Furthermore, our analysis is limited to four countries surveyed in the STEP program that implemented a compulsory schooling reform affecting individuals born between 1947 and 1989.\footnote{There are some more countries contained in the STEP program that also experienced compulsory schooling reforms, but these reforms took place in the late 1990s or early 2000s.} While we do observe some individuals born after 1989, many of them likely had not completed their education at the time of data collection, making them incomparable to older individuals who had already finished their education.

When interpreting the results, it is important to keep in mind that we estimate an overall effect across a diverse set of countries and reforms. Although all countries are low- and middle-income countries, they are spread across three different continents. Moreover, the reforms vary in scope, affecting primary and lower-secondary education, with the number of mandatory years ranging from three to six years. However, it is reassuring that our results are not driven by any single reform (Figure \ref{fig:leave_one_out}) and that the estimates from single reforms are, even though noisy, predominantly pointing in the same direction (Table \ref{tab:single_reforms}). Due to the limited number of observations per country, we are unable to study individual reforms in isolation.

The limited number of observations presents two additional drawbacks. First, it prevents us from restricting the sample to individuals born just before or just after the first affected cohort, forcing us to rely on estimating cohort trends. Second, it limits our ability to conduct heterogeneity analyses, which could have provided insights into how treatment effects vary across gender or socio-economic background. Despite these limitations, our study is novel in estimating the causal effect of compulsory education reforms on multiple non-cognitive skills.
