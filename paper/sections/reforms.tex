\begin{table}[htbp]
	\caption{Overview of compulsory schooling reforms}
	\label{tab:reforms}
	\small
	\centering
	\begin{threeparttable}
	\begin{tabular}{l p{1.3cm} p{2.5cm} p{2.2cm} p{1.2cm} p{2cm}}
		\hline\hline
		\textbf{Country} & \textbf{Year of reform} & \textbf{Compulsory stage} & \textbf{Change in compulsory years} & \textbf{Pivotal cohort} & \textbf{Gradual implementation} \\
		\hline
		Ghana & 1961  & primary & 0 to 6 & 1954 & No \\
		Ghana & 1987 & lower secondary & 6 to 9 & 1975 & No \\
		Colombia & 1991 & lower secondary & 5 to 9 & 1981 & No \\
		Vietnam & 1991 & primary & 0 to 5 & 1977 & Yes \\
		Bolivia & 1994 & lower secondary & 5 to 8 & 1983 & No \\
		\hline\hline
	\end{tabular}
	\begin{tablenotes}
		\footnotesize
		\item \textit{Notes:} Information on compulsory schooling reforms is drawn from scientific publications, UNESCO reports, legal texts, and official government reports. We analyze these five listed reforms across four countries.
	\end{tablenotes}
\end{threeparttable}

\end{table}

We use information on compulsory schooling reforms mainly from scientific publications, UNESCO reports, legal texts, and official government reports. Since treatment is defined based on birth cohorts affected and not affected by the reforms, we focus on compulsory schooling reforms that are relevant to individuals born after 1947 and before 1989. As a result, we analyze five reforms across four countries: Ghana (1961 and 1987), Colombia (1991), Vietnam (1991), and Bolivia (1994). Table \ref{tab:reforms} provides an overview of these reforms.\footnote{
	Besides the five reforms described above, there was a sixth reform within the time-frame we consider for our analysis. This sixth reform was in China in 1986, introduced nine years of mandatory education, and was implemented gradually within the Chinese provinces. \citet{jiang_education_2020} and \citet{huang_understanding_2015} investigate the reform with data from whole China. However, the regional development in China was very unbalanced resulting in a major challenge for China to popularize the nine years compulsory education in all regions \parencite{wang_schooling_2020}. In 1995, Yunnan province was categorized as one of nine economically underdeveloped provincial-level regions with greatest difficulties in the implementation of compulsory education \parencite{wang_schooling_2020}. Given that we only have data for Yunnan province and not whole China we decided to exclude this province from our analysis.
} Two out of the five reforms introduced compulsory primary education, while the other three extended compulsory schooling by mandating lower secondary education. The pivotal cohort is either taken from published research or calculated based on the reform year and the typical school entry age. We assume that children who were already eligible to leave school under the previous regulations were not required to return after the reform. While little is known about the strictness of enforcement, evidence suggests that despite potentially weak implementation, these reforms successfully increased school enrollment.

A more detailed description of the reforms and their potential confounding factors is given below. On the one hand, it might be considered concerning to pool effects from different reforms. However, if similar effects are found across all these various reforms---which all share the characteristic of increasing the number of compulsory school years---this can also be reassuring in that the observed effect is truly due to the additional years of schooling rather than potential confounders. By this means, we provide estimates from each reform separately in Table \ref{tab:single_reforms}. These are more noisy due to the sample sizes but still informative.

\paragraph{Ghana 1961} \label{sec:ghana1} \mbox{}\\
In 1961, Ghana's education reform took effect, introducing free, universal, and compulsory basic education for six years. Even before this reform, Ghana was recognized as having the most developed education system in Africa. The government's primary objective was to eradicate illiteracy. A decade earlier, in 1951, the ``Accelerated Development Plans'' (ADP) for education were launched, leading to a rapid and steady growth in the number of schools. By 1961, the ADP had also significantly increased the number of teachers, and new teacher training colleges were established in 1961. As a result, primary school enrollment increased from 153,360 in 1951 to 1,137,495 in 1966 \parencite{akyeampong_educational_2007}.

While school construction and the expansion of the teaching workforce could be seen as confounding factors, we consider them indicators of continued educational quality, despite the sharp rise in student enrollment. As \citet[p.~5]{kadingdi_policy_2006} notes, ``even though school enrollments increased following the 1961 Education Act, the quality of teaching and learning appears to have remained the same.''


\paragraph{Ghana 1987} \label{sec:ghana2} \mbox{}\\
Between the 1961 and 1981 reforms, Ghana experienced political instability, marked by multiple military governments. In December 1981, another military government took power. These instabilities led to a deteriorating education system and stagnating school enrollment rates by 1983. Consequently, the new government prioritized a reform of the education system, seeking World Bank credits and grants to finance it \parencite{kadingdi_policy_2006}.

The 1987 reform extended compulsory schooling from six to nine years, introducing a mandatory three-year junior secondary education. Access to education was further improved through infrastructure expansion \parencite{kadingdi_policy_2006}. Additionally, the government aimed to make the curriculum more relevant to Ghana's socio-economic conditions \parencite{kadingdi_policy_2006}, shifting the focus from grammar-based education to vocational and technical training \parencite{akyeampong_educational_2007}.

This shift in curriculum could have influenced non-cognitive skills, making it difficult to isolate the impact of additional years of schooling from changes in educational content. However, the reform did not significantly increase the number of technically and vocationally trained workers. As \citet{akyeampong_educational_2007} explains, formal schooling proved ineffective in changing attitudes toward vocational and technical education in particular and employment in general \parencite{foster_education_1965, king_vocational_2002}. Thus, ``the 1987 education reform had been far from making an impact on the labour market profile'' \parencite[][p.~6]{akyeampong_educational_2007} in Ghana.


\paragraph{Colombia 1991} \label{sec:colombia} \mbox{}\\
In 1991, the Colombian government increased mandatory schooling from five to nine years, making the 1981 birth cohort the pivotal group \parencite{urbina_mass_2022}. The primary goal was to improve primary and lower secondary school attainment, particularly among students at risk of dropping out \parencite{unesco_situacion_2001}. To accommodate the expected rise in enrollment, the reform was accompanied by an ambitious school construction plan.

The reform successfully boosted secondary school enrollment: between 1993 and 2004, enrollment rates among 12- to 17-year-olds increased from 68\% to 78\% \parencite{urbina_mass_2022, unesco_educacion_2004}. Additionally, in 1991, Colombia enacted a new constitution aimed at resolving conflicts between the government, paramilitary groups, and guerrilla forces. Since this political change affected all individuals, not just those born in 1981 or later, we do not consider it a confounding factor in our analysis.

\paragraph{Vietnam 1991} \label{sec:vietnam} \mbox{}\\
The 1991 school reform in Vietnam introduced mandatory five-year primary education to raise overall education levels. As a result, net primary school enrollment increased from 86.0\% in 1990-91 to 96.7\% in 1997-98 \parencite{national_committee_for_efa_assessment_assessment_1999}. However, ``the implementation of the reform was `piecemeal rather than comprehensive' across the country, requiring years of preparation'' \parencite[][p. 3]{cornelissen_multigenerational_2022}. Therefore, \citet{cornelissen_multigenerational_2022} consider only individuals born four years after the first-affected cohort (1977) as fully exposed to the reform. Our data confirm that only four years after the pivotal cohort, the reform's impact on years of education becomes clearly positive and comparable in magnitude to the other reforms under investigation. Consequently, we follow \citet{cornelissen_multigenerational_2022} and distinguish between fully and partially treated individuals in Vietnam. Fully treated individuals are those born in 1981 or later.

Beyond extending compulsory education, the reform increased investments in education, funding the construction of new schools, teacher training, and financial aid for disadvantaged students \parencite{nguyen_trends_2004, cornelissen_multigenerational_2022}. These measures aimed to boost enrollment rates, particularly among disadvantaged children.


\paragraph{Bolivia 1994} \label{sec:bolivia} \mbox{}\\
In 1994, compulsory education in Bolivia was extended from five to eight years, first affecting the 1983 birth cohort \parencite{urbina_mass_2022}. As in Colombia, the reform aimed to increase primary and lower secondary school attainment, particularly among vulnerable students at risk of dropping out \parencite{unesco_situacion_2001}. Enrollment rates were particularly low among indigenous children and those from rural areas. The reform was successful in increasing enrollment, with sixth-grade promotion rates rising from 52\% in 1994 to 85\% in 2001 \parencite{contreras_bolivian_2003}.

Beyond expanding compulsory schooling, the reform introduced a multicultural curriculum, making school materials available in multiple languages and incorporating folk culture elements. Since the reform, teachers have been trained in bilingual and intercultural education. This curriculum change may have influenced non-cognitive skills, for example, potentially enhancing openness to experience through exposure to diverse cultural perspectives. To address potential reform-specific confounding factors, we will assess the robustness of our results by estimating the reform effects in sub-samples that systematically exclude one reform at a time. This approach ensures that the estimated effects of compulsory schooling reforms are not driven by a single, potentially confounded schooling reform.
