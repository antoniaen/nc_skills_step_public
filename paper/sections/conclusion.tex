
The primary aim of education is often to enhance cognitive skills that contribute to better life outcomes. However, non-cognitive skills also play a crucial role in shaping labor market success. This paper investigates whether compulsory education reforms can strengthen non-cognitive skills and thereby improve individuals' labor market prospects. Leveraging within-country variation in years of compulsory education across four low- and middle-income countries, we estimate the causal effects of these reforms on non-cognitive skills.

Our findings reveal that extending mandatory education significantly affects a range of non-cognitive skills, including personality traits, economic preferences, decision-making patterns, grit, and hostile attribution bias. Importantly, we find no effect on employment probabilities or transitions from self- or family-employment to wage work, indicating that changes in these labor market outcomes do not drive our results.

To evaluate the impact of changes in non-cognitive skills on earning potential, we estimate augmented Mincer wage regressions. Our results suggest that reform induced changes in non-cognitive skills positively contribute to wages. Accounting for the effects on years of education and cognitive skills, we predict an overall wage increase of 9\% based on the treatment effects of the reforms and Mincer wage regressions.

These findings have several important implications, highlighting that non-cognitive skills---like cognitive skills---can be shaped through mandatory education. Given that these skills are rewarded in the labor market, our results highlight the importance of designing education systems that effectively enhance these skills. Our results suggest that education policies should incorporate strategies to develop personality traits, preferences, foster traits like grit, decision-making patterns, and preferences, while mitigating hostile attribution bias. The relative importance of specific non-cognitive skills may vary across contexts, suggesting that education policies aimed at enhancing productivity and earnings should be tailored to the skill demands of each labor market, especially when comparing high-income and low- to middle-income countries.


We also explore potential mechanisms beyond years of education through which compulsory education reforms might influence non-cognitive skills, such as improvement in education quality, shifts in the perceived value of education, and changes in peer ability composition. While we cannot rule out the influence of other factors, our analysis suggests that these alternative mechanisms are unlikely to be the primary drivers of the observed effects.

However, the returns to non-cognitive skills in the countries we studied differ from those in high-income countries. Additionally, the effect of education on patience in our sample contrasts with findings from developed countries, suggesting that our results may not generalize to high-income settings. Future research should investigate whether education enhances non-cognitive skills in developed countries, especially as the growing impact of artificial intelligence may reduce the demand for certain cognitive skills while amplifying the value of certain non-cognitive skills.
