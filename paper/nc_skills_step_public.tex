\documentclass[11pt, a4paper, leqno]{article}
\usepackage{a4wide}
\usepackage[T1]{fontenc}
\usepackage[utf8]{inputenc}
\usepackage{float, afterpage, rotating, graphicx}
\usepackage{epstopdf}
\usepackage{longtable, booktabs, tabularx}
\usepackage{fancyvrb, moreverb, relsize}
\usepackage{eurosym, calc}
\usepackage{amsmath, amssymb, amsfonts, amsthm, bm}
\usepackage{caption}
\usepackage{mdwlist}
\usepackage{xfrac}
\usepackage{setspace}
\usepackage[dvipsnames]{xcolor}
\usepackage{subcaption}
\usepackage{minibox}
\usepackage[flushleft]{threeparttable} % for tablenotes
\usepackage{multirow} % tables
\usepackage{dsfont} % for bold 1 in math
\usepackage{tikz} % Graphs
\usepackage{xcolor}
\usetikzlibrary{decorations.pathreplacing,calligraphy} % Curly calligraphic braces in tikzpicture


\usepackage[
    natbib=true,
    style=apa,
    maxcitenames=3,
    useprefix=false,
    sortcites=true,
    backend=biber,
    date=year,
    isbn=false,url=false,eprint=false,
    labeldateparts=true,
    uniquelist=minyear
]{biblatex}
\AtBeginDocument{\toggletrue{blx@useprefix}}
\AtBeginBibliography{\togglefalse{blx@useprefix}}
\setlength{\bibitemsep}{1.5ex}
\addbibresource{Schooling_manually_capitalized_version.bib}

\AtEveryBibitem{
	\clearlist{publisher}
	\clearlist{location}
	\clearfield{note}
	\clearfield{howpublished}
}



\usepackage[unicode=true]{hyperref}
\hypersetup{
    colorlinks=true,
    linkcolor=black,
    anchorcolor=black,
    citecolor=NavyBlue,
    filecolor=black,
    menucolor=black,
    runcolor=black,
    urlcolor=NavyBlue
}


\widowpenalty=10000
\clubpenalty=10000

\setlength{\parskip}{1ex}
\setlength{\parindent}{0ex}
\setstretch{1.5}

% JEL codes command
\providecommand{\JEL}[1]
{
	\small
	\textbf{JEL codes:} #1
}
% Keywords command
\providecommand{\keywords}[1]
{
	\small
	\textbf{Keywords:} #1
}


% Manually set \thanks symbols
\makeatletter
\renewcommand*{\@fnsymbol}[1]{%
	\ensuremath{%
		\ifcase#1 \or \star \or * \else \@alph{\numexpr#1-2\relax} \fi%
	}%
}
\makeatother


\begin{document}

\title{The effect of compulsory education on non-cognitive skills: Evidence from low- and middle-income countries\thanks{We would like to thank Lucia Gomez Llactahuamani for her excellent research assistance. Funding by the Deutsche Forschungsgemeinschaft (DFG, German Research Foundation) through CRC TR 224 (Project A05) and Germany’s Excellence Strategy – EXC 2126/1–390838866 as well as through NSF Award 2201888 is gratefully acknowledged.}}

\author{Antonia K. Entorf\thanks{Corresponding author. E-mail address: antonia.entorf@uni-bonn.de} \thanks{University of Bonn, Regina-Pacis-Weg 3, 53113 Bonn, Germany} \thanks{ECONtribute, Niebuhrstraße 5, 53113 Bonn, Germany} \and Thomas J. Dohmen\footnotemark[3] \footnotemark[4] \thanks{Maastricht University, Tongersestraat 53, 6211 LM Maastricht, Netherlands} \thanks{IZA Institute of Labor Economics, Schaumburg-Lippe-Straße 5-9, 53113 Bonn, Germany}}

\date{\today}

\maketitle


\begin{abstract}
    Personality traits, preferences, and attitudes significantly influence labor market outcomes, and these non-cognitive skills are shaped by the social environment. While curriculum interventions can impact these skills, the effect of compulsory education on non-cognitive skills is less well understood. This study investigates the impact of extending compulsory education by examining educational reforms in four low- and middle-income countries. Utilizing cross-sectional data from the World Bank's 2012/2013 initiative, we analyze the within-country variation in compulsory education years. Our findings indicate that increased compulsory education decreases emotional stability, grit, hostile attribution bias, patience, and willingness to take risks, while enhancing openness to experience and alternative solution or consequential thinking.
\end{abstract}


%TC:ignore
\JEL{J24, I20, I26, D91}

\keywords{Non-cognitive skills, Education, Wage returns, Personality, Economic preferences}
%TC:endignore

\clearpage

\section{Introduction} % (fold) \label{sec:introduction}

While numerous studies establish positive effects of education on cognitive skills \parencite[e.g.][]{schneeweis_does_2014, dahmann_how_2017, cappellari_effects_2023}, much less is known about the effect of education on personality traits, preferences, and attitudes. These personal attributes are often referred to as non-cognitive skills because they are rewarded in the labor market \parencite{bowles_incentive-enhancing_2001} similar to cognitive skills. With technological advancements potentially substituting human cognitive abilities and the increasing complexity of organizing work, the importance of non-cognitive skills is expected to grow, leading to higher returns for these skills \parencite{deming_growing_2017, edin_rising_2022}. Non-cognitive skills, along with cognitive skills, significantly influence socioeconomic outcomes such as employment, wages, and health \parencite{heckman_effects_2006, heckman_understanding_2013}, including in low- and middle-income countries \parencite{sharma_does_2018, gertler_labor_2014, glewwe_cognitive_2017}. Despite being similarly predictive of socioeconomic outcomes as cognitive skills, and increasingly important in the labor market, it remains an open question whether compulsory education affects non-cognitive skills. We cannot simply infer findings from studies on cognitive skills to non-cognitive skills as these attributes may respond differently to educational inputs. Previous research, such as the studies by \citet{heckman_understanding_2013} and \citet{alan_ever_2019}, indicates that non-cognitive skills are affected by early childhood programs and curriculum interventions, therefore, one could expect that compulsory education has an impact as well. We aim to investigate whether this is indeed the case and how various non-cognitive skills are affected. Analyzing this relationship is crucial, as it would provide insights into whether and how education systems can nurture traits that are increasingly valuable in the economy.


In this paper, we analyze the long-term effects of compulsory schooling reforms on non-cognitive skills.\footnote{
	In Table \ref{tab:res_lit_test} in the Appendix section with findings for additional outcomes, section \ref{sec:app_add_outcomes}, we also report on the effect of compulsory schooling reforms on cognitive skills.
} We examine five educational reforms that mandated compulsory primary or lower secondary education in four low- and middle-income countries: Bolivia, Colombia, Ghana, and Vietnam. Our analysis relies on cross-sectional World Bank data from 2012/13, which is representative of individuals aged 15-64, living in urban regions of these countries. The dataset includes measures of the Big Five personality traits, risk and time preferences, grit, hostile attribution bias, and decision-making patterns. Moreover, scores from a literacy test are included. For causal identification of effects, the cohort trends in these skills must be well approximated and no other event should differently affect treatment and control groups. We assess the returns to non-cognitive and cognitive skills by estimating augmented Mincer Wage regressions.

We find that the reforms expanding compulsory education affected non-cognitive skills. Specifically, the reforms decreased emotional stability, grit, hostile attribution bias, willingness to take risks, and patience, while increasing openness to experience and alternative solution or consequential thinking. In our sample, increased openness to experience and reduced hostile attribution bias are positively related to wages, whereas a reduced willingness to take risks is negatively associated with wages. The other skills affected by the reforms do not show a significant relationship with wages. Furthermore, we find no evidence that individuals affected by the reforms differ in their probability or type of employment compared to those not affected. Considering the effects on years of education and cognitive skills, our correlational results from the augmented Mincer wage regression suggest that educational expansions, on average, increase wages by 9\%, with 40\% of this predicted increase stemming from changes in non-cognitive skills. We discuss potential mechanisms---such as educational quality, perceived importance of education, and ability mixing---to better understand how compulsory schooling reforms might influence the development of non-cognitive skills.

To the best of our knowledge, this is the first study to examine the causal effect of reforms raising years of compulsory education on personality traits, grit, hostile attribution bias, and decision-making patterns. Our study complements existing research that has documented the causal effects of curriculum interventions on specific character traits. For example, \citet{alan_ever_2019} and \citet{santos_can_2022} show that school interventions aimed at enhancing children's grit can be successful. Other related studies investigate the effects of different school types, such as a selective college or vocational versus general secondary education \parencite{dasgupta_effects_2022, ollikainen_effect_2024} on personality, or the effects of special educational interventions, such as a German school reform or an equalization policy in Korea that resulted in ability mixing within classes \parencite{dahmann_impact_2014, dahmann_cross-fertilizing_2018, ahn_long-term_2021}. For instance, \citet{dasgupta_effects_2022} find that attending a selective college decreases men's extraversion and conscientiousness, while \citet{dahmann_cross-fertilizing_2018} find that the introduction of G8 in Germany (a reduction in the years of schooling while keeping the same curriculum, which led to an increased intensity of schooling) decreased individuals' emotional stability.


Furthermore, this study contributes to the literature on the effect of education on economic preferences, which has yielded mixed findings. \citet{jung_does_2015} finds that a school reform increasing compulsory education in Great Britain decreased individuals' willingness to take risk, whereas \citet{tawiah_does_2022} reports that a similar reform in West Germany increased the willingness to take risk. Our study is the first to provide evidence on the effect of education on willingness to take risks in low- and middle-income countries, and our finding of a negative effect aligns with \citeauthor{jung_does_2015}'s \parencite*{jung_does_2015} results for Great-Britain.

Regarding time preferences, three studies provide relevant empirical findings. \citet{dohmen_effect_2022} show that compulsory schooling reforms positively affect patience across 48 countries, with this overall positive effect driven by developed countries, while there is a null effect for developing countries. Similarly, \citet{jung_does_2021} find a positive effect on patience, identifying this effect through increased years of education induced by reduced schooling costs in Indonesia. In contrast, \citet{tawiah_does_2022} finds a negative effect on patience.

Finally, we contribute to the literature on labor market returns to non-cognitive and cognitive skills.\footnote{
	We also show results for the effect on cognitive skills, in particular, on literacy test scores (Table \ref{tab:res_lit_test}). Thereby, we contribute to the literature on the causal effect of compulsory schooling reforms on cognitive skills in non-OECD countries. While numerous studies from the fields of psychology and economics investigate the effect of schooling on cognitive skills, they usually provide evidence from OECD countries \parencite[see, for instance,][]{schneeweis_does_2014, cappellari_effects_2023}.
} For instance, \citet{heckman_understanding_2013} study how the Perry Preschool program for disadvantaged African Americans boosted individuals' labor market outcomes, showing short-term effects on IQ and long-term effects on externalizing behavior and academic motivation. \citet{heineck_returns_2010} investigate the wage-returns of the Big-Five personality traits, locus of control, reciprocity, and fluid intelligence using data from Germany. \citet{hanushek_coping_2017} study cross-country differences in returns to cognitive (numeracy) skills using data from the PIAAC survey. \citet{piopiunik_skills_2020} use fictitious resumes to study which skills improve employability. \citet{heckman_effects_2006} analyze the effects of both cognitive and non-cognitive skills, modeling both types of skills as latent traits, and study the effect on wages, work experience, and occupational choice. They find that both types of skills significantly impact labor market outcomes.


The paper is structured as follows. Section \ref{sec:reforms} provides a detailed overview about the compulsory schooling reforms. Section \ref{sec:data} introduces the data used in our analysis. Section \ref{sec:emp_strategy} outlines the research design, the empirical approach, and the effects we expect to observe based on the literature. Section \ref{sec:results} presents the results, provides robustness checks of the results, and evaluates the findings based on estimated wage returns to skills. Section \ref{sec:discussion} examines the potential mechanisms through which expanding compulsory education could impact non-cognitive skills and discusses the limitations of the analysis. Section \ref{sec:conclusion} concludes and considers the implications of our findings for the malleability of non-cognitive skills.


\section{Compulsory schooling reforms} \label{sec:reforms}
\begin{table}[htbp]
	\caption{Overview of compulsory schooling reforms}
	\label{tab:reforms}
	\small
	\centering
	\begin{threeparttable}
	\begin{tabular}{l p{1.3cm} p{2.5cm} p{2.2cm} p{1.2cm} p{2cm}}
		\hline\hline
		\textbf{Country} & \textbf{Year of reform} & \textbf{Compulsory stage} & \textbf{Change in compulsory years} & \textbf{Pivotal cohort} & \textbf{Gradual implementation} \\
		\hline
		Ghana & 1961  & primary & 0 to 6 & 1954 & No \\
		Ghana & 1987 & lower secondary & 6 to 9 & 1975 & No \\
		Colombia & 1991 & lower secondary & 5 to 9 & 1981 & No \\
		Vietnam & 1991 & primary & 0 to 5 & 1977 & Yes \\
		Bolivia & 1994 & lower secondary & 5 to 8 & 1983 & No \\
		\hline\hline
	\end{tabular}
	\begin{tablenotes}
		\footnotesize
		\item \textit{Notes:} Information on compulsory schooling reforms is drawn from scientific publications, UNESCO reports, legal texts, and official government reports. We analyze these five listed reforms across four countries.
	\end{tablenotes}
\end{threeparttable}

\end{table}

We use information on compulsory schooling reforms mainly from scientific publications, UNESCO reports, legal texts, and official government reports. Since treatment is defined based on birth cohorts affected and not affected by the reforms, we focus on compulsory schooling reforms that are relevant to individuals born after 1947 and before 1989. As a result, we analyze five reforms across four countries: Ghana (1961 and 1987), Colombia (1991), Vietnam (1991), and Bolivia (1994). Table \ref{tab:reforms} provides an overview of these reforms.\footnote{
	Besides the five reforms described above, there was a sixth reform within the time-frame we consider for our analysis. This sixth reform was in China in 1986, introduced nine years of mandatory education, and was implemented gradually within the Chinese provinces. \citet{jiang_education_2020} and \citet{huang_understanding_2015} investigate the reform with data from whole China. However, the regional development in China was very unbalanced resulting in a major challenge for China to popularize the nine years compulsory education in all regions \parencite{wang_schooling_2020}. In 1995, Yunnan province was categorized as one of nine economically underdeveloped provincial-level regions with greatest difficulties in the implementation of compulsory education \parencite{wang_schooling_2020}. Given that we only have data for Yunnan province and not whole China we decided to exclude this province from our analysis.
} Two out of the five reforms introduced compulsory primary education, while the other three extended compulsory schooling by mandating lower secondary education. The pivotal cohort is either taken from published research or calculated based on the reform year and the typical school entry age. We assume that children who were already eligible to leave school under the previous regulations were not required to return after the reform. While little is known about the strictness of enforcement, evidence suggests that despite potentially weak implementation, these reforms successfully increased school enrollment.

A more detailed description of the reforms and their potential confounding factors is given below. On the one hand, it might be considered concerning to pool effects from different reforms. However, if similar effects are found across all these various reforms---which all share the characteristic of increasing the number of compulsory school years---this can also be reassuring in that the observed effect is truly due to the additional years of schooling rather than potential confounders. By this means, we provide estimates from each reform separately in Table \ref{tab:single_reforms}. These are more noisy due to the sample sizes but still informative.

\paragraph{Ghana 1961} \label{sec:ghana1} \mbox{}\\
In 1961, Ghana's education reform took effect, introducing free, universal, and compulsory basic education for six years. Even before this reform, Ghana was recognized as having the most developed education system in Africa. The government's primary objective was to eradicate illiteracy. A decade earlier, in 1951, the ``Accelerated Development Plans'' (ADP) for education were launched, leading to a rapid and steady growth in the number of schools. By 1961, the ADP had also significantly increased the number of teachers, and new teacher training colleges were established in 1961. As a result, primary school enrollment increased from 153,360 in 1951 to 1,137,495 in 1966 \parencite{akyeampong_educational_2007}.

While school construction and the expansion of the teaching workforce could be seen as confounding factors, we consider them indicators of continued educational quality, despite the sharp rise in student enrollment. As \citet[p.~5]{kadingdi_policy_2006} notes, ``even though school enrollments increased following the 1961 Education Act, the quality of teaching and learning appears to have remained the same.''


\paragraph{Ghana 1987} \label{sec:ghana2} \mbox{}\\
Between the 1961 and 1981 reforms, Ghana experienced political instability, marked by multiple military governments. In December 1981, another military government took power. These instabilities led to a deteriorating education system and stagnating school enrollment rates by 1983. Consequently, the new government prioritized a reform of the education system, seeking World Bank credits and grants to finance it \parencite{kadingdi_policy_2006}.

The 1987 reform extended compulsory schooling from six to nine years, introducing a mandatory three-year junior secondary education. Access to education was further improved through infrastructure expansion \parencite{kadingdi_policy_2006}. Additionally, the government aimed to make the curriculum more relevant to Ghana's socio-economic conditions \parencite{kadingdi_policy_2006}, shifting the focus from grammar-based education to vocational and technical training \parencite{akyeampong_educational_2007}.

This shift in curriculum could have influenced non-cognitive skills, making it difficult to isolate the impact of additional years of schooling from changes in educational content. However, the reform did not significantly increase the number of technically and vocationally trained workers. As \citet{akyeampong_educational_2007} explains, formal schooling proved ineffective in changing attitudes toward vocational and technical education in particular and employment in general \parencite{foster_education_1965, king_vocational_2002}. Thus, ``the 1987 education reform had been far from making an impact on the labour market profile'' \parencite[][p.~6]{akyeampong_educational_2007} in Ghana.


\paragraph{Colombia 1991} \label{sec:colombia} \mbox{}\\
In 1991, the Colombian government increased mandatory schooling from five to nine years, making the 1981 birth cohort the pivotal group \parencite{urbina_mass_2022}. The primary goal was to improve primary and lower secondary school attainment, particularly among students at risk of dropping out \parencite{unesco_situacion_2001}. To accommodate the expected rise in enrollment, the reform was accompanied by an ambitious school construction plan.

The reform successfully boosted secondary school enrollment: between 1993 and 2004, enrollment rates among 12- to 17-year-olds increased from 68\% to 78\% \parencite{urbina_mass_2022, unesco_educacion_2004}. Additionally, in 1991, Colombia enacted a new constitution aimed at resolving conflicts between the government, paramilitary groups, and guerrilla forces. Since this political change affected all individuals, not just those born in 1981 or later, we do not consider it a confounding factor in our analysis.

\paragraph{Vietnam 1991} \label{sec:vietnam} \mbox{}\\
The 1991 school reform in Vietnam introduced mandatory five-year primary education to raise overall education levels. As a result, net primary school enrollment increased from 86.0\% in 1990-91 to 96.7\% in 1997-98 \parencite{national_committee_for_efa_assessment_assessment_1999}. However, ``the implementation of the reform was `piecemeal rather than comprehensive' across the country, requiring years of preparation'' \parencite[][p. 3]{cornelissen_multigenerational_2022}. Therefore, \citet{cornelissen_multigenerational_2022} consider only individuals born four years after the first-affected cohort (1977) as fully exposed to the reform. Our data confirm that only four years after the pivotal cohort, the reform's impact on years of education becomes clearly positive and comparable in magnitude to the other reforms under investigation. Consequently, we follow \citet{cornelissen_multigenerational_2022} and distinguish between fully and partially treated individuals in Vietnam. Fully treated individuals are those born in 1981 or later.

Beyond extending compulsory education, the reform increased investments in education, funding the construction of new schools, teacher training, and financial aid for disadvantaged students \parencite{nguyen_trends_2004, cornelissen_multigenerational_2022}. These measures aimed to boost enrollment rates, particularly among disadvantaged children.


\paragraph{Bolivia 1994} \label{sec:bolivia} \mbox{}\\
In 1994, compulsory education in Bolivia was extended from five to eight years, first affecting the 1983 birth cohort \parencite{urbina_mass_2022}. As in Colombia, the reform aimed to increase primary and lower secondary school attainment, particularly among vulnerable students at risk of dropping out \parencite{unesco_situacion_2001}. Enrollment rates were particularly low among indigenous children and those from rural areas. The reform was successful in increasing enrollment, with sixth-grade promotion rates rising from 52\% in 1994 to 85\% in 2001 \parencite{contreras_bolivian_2003}.

Beyond expanding compulsory schooling, the reform introduced a multicultural curriculum, making school materials available in multiple languages and incorporating folk culture elements. Since the reform, teachers have been trained in bilingual and intercultural education. This curriculum change may have influenced non-cognitive skills, for example, potentially enhancing openness to experience through exposure to diverse cultural perspectives. To address potential reform-specific confounding factors, we will assess the robustness of our results by estimating the reform effects in sub-samples that systematically exclude one reform at a time. This approach ensures that the estimated effects of compulsory schooling reforms are not driven by a single, potentially confounded schooling reform.


\section{Data} \label{sec:data}
\subsection{STEP Program}

The data used in this study are drawn from the World Bank’s Skills Towards Employability and Productivity (STEP) Measurement Program, the first initiative to systematically measure skills in low- and middle-income countries. The program aims to provide policy-relevant data to better understand labor market skill requirements. The STEP program includes both a household-based and an employer-based survey. For our analysis, we use data from the household survey.

The household-survey introduces a distinct module that makes the data particularly suitable for our analysis. It contains an extensive array of non-cognitive skills measures. The survey items used to measure personality and behaviors were carefully designed and selected by a multidisciplinary team of social scientists, including economists and psychologists. The guiding principles of the selection process included ``the applicability and comprehension of the items in low-literacy cultures where people have little or no experience answering self-reported questions'' \parencite[][p. 29]{pierre_step_2014}. Moreover, the items were designed brief, based on prior evidence of scale reliability and validity, and on prior evidence of predictive validity. Besides measures of non-cognitive skills, the survey contains a direct assessment of reading proficiency developed by the Educational Testing Service (ETS). The assessment is scored on the same scale as the OECD's PIAAC data. Since cognitive skills are not the focus of our analysis, we only provide an analysis of these in the Appendix section \ref{sec:app_add_outcomes}.

The STEP Program currently includes household data from thirteen countries: Armenia, Bolivia, Colombia, Georgia, Ghana, Kenya, Laos, Sri Lanka, Macedonia, Philippines, Ukraine, Vietnam, and  Yunnan Province (China).\footnote{The World Bank's plan is to add more countries.} Data collection occurred in 2012 and 2013, except for the Philippines, where a different survey was conducted in 2015. To ensure data comparability, we exclude the Philippines from our analysis. Our study focuses on Bolivia, Colombia, Ghana, and Vietnam, as these countries implemented relevant compulsory education reforms during the period of interest.\footnote{\citet{world_bank_bolivia_2012}, \citet{world_bank_colombia_2012}, \citet{world_bank_ghana_2013}, \citet{world_bank_vietnam_2012}.} In all surveyed countries, the target population consists of urban residents aged 15 to 64.


\subsection{Measures of non-cognitive skills}
The STEP data includes measures of personality, behavior, and preferences. The personality measures are based on the Big Five taxonomy of personality traits, which includes agreeableness, conscientiousness, extraversion, openness to experience, and emotional stability. These dimensions have been derived from a factor analysis of a broad personality inventory and have been replicated across cultures \parencite{allport_trait-names_1936, john_big_1999, almlund_chapter_2011}.

The STEP data includes measures of three key behaviors: grit, hostile attribution bias, and decision-making patterns. Grit captures perseverance and passion for long-term goals \parencite{duckworth_grit_2007}. Although related to conscientiousness, grit has demonstrated predictive validity beyond this personality trait. Hostile attribution bias refers to the tendency to perceive hostile intent in others' behavior, even when it is ambiguous or benign, and is associated with aggressive behavior \parencite{dodge_social_2003, dodge_translational_2006}. Decision-making patterns are measured through alternative solution thinking and consequential thinking.

The former captures the ability to consider multiple options when making decisions, while the latter captures the tendency to consider future consequences for oneself or others. These decision-making patterns are evaluated using four survey questions: \textit{Do you think about how the things you do will affect you in the future?, Do you think carefully before you make an important decision?, Do you ask for help when you don’t understand something?, Do you think about how the things you will do will affect others?}



In summary, the household questionnaire includes three items for each personality trait based on the short Big Five Inventory by \citet{john_big_1999} \parencite[see][for a validation]{lang_short_2011}, three items from the Grit Scale \parencite{duckworth_grit_2007}, two items assessing hostile attribution bias, and four items measuring decision-making patterns \parencite{mann_melbourne_1997}. While the items for grit, hostile attribution bias, and decision-making patterns represent subsets of the original scales, their selection was guided by expert input: The developers of the Grit Scale (Angela Duckworth) and the hostile attribution bias scale (Kenneth Dodge) were consulted during the STEP survey design process. A pilot study informed the adoption of a four-point scale, ranging from \textit{almost always} to \textit{almost never}.

The STEP data, further, includes a hypothetical lottery choice between a safe and a risky option (\textit{Get \$120 for sure or flip a coin for \$0 or \$360}) and a hypothetical choice between a smaller amount of money today versus a larger amount of money next year (\textit{Receive \$1200 today or \$1800 in one year}).\footnote{
	There was a mistake in the questionnaire for Colombia in the items measuring the time preference (3.240.000 Pesos have been switched with 3.456.000 Pesos). Therefore, instead of the planned measure based on three hypothetical choices, we only make use of the first out of the three questions and create binary measures for economic preferences. The results are qualitatively the same when we use the measures based on all three questions and exclude Colombia for estimating the effect on the time preference.
} \citet{falk_preference_2023} showed that measures from hypothetical survey questions can predict real choices in corresponding, incentivized experiments. Based on these hypothetical choices, we construct measures for risk and time preferences as dummy variables that indicate whether the risky/patient option was chosen.

The measures used in this analysis are generated by standardizing the average score from the dimension-specific questions on Big Five personality traits, grit, hostile attribution bias, and decision-making patterns. Time and risk preferences are captured by a dummy variable indicating whether an individual chooses the patient or risky option, respectively. Following \citet{laajaj_challenges_2019}, we correct the STEP data for acquiescence response bias, which refers to an individual's tendency to generally agree or disagree with survey items. This correction is possible because some questions are reverse-coded, For example, \textit{``Are you relaxed during stressful situations?''} and \textit{``Do you get nervous easily?''} both assess emotional stability, but the second question is reverse-coded. We apply the correction method outlined in the Materials and Methods section of \citet{laajaj_challenges_2019} to adjust the measures of personality traits, grit, hostile attribution bias, and decision-making patterns for this bias.\footnote{As additional information on the final non-cognitive skill measures, we provide pairwise Pearson correlation coefficients between these and individual characteristics, e.g. age or gender, together with the correlations' signs found in the literature in Table \ref{tab:correlations}.}



\subsection{Estimation sample}

To create the estimation sample, we classify individuals in our sample as treated or control individuals based on their year of birth, country, and the first affected birth cohort of the respective reform, which we refer to as the pivotal cohort. Individuals born during the pivotal cohort year or later are assigned to the treatment group, while those born before the pivotal cohort are assigned to the control group:

{\centering
	$ \displaystyle
	\begin{aligned}
		D_{i, country} = \mathds{1}_{\{\text{birth-year}_{i, country} \text{ } \geq \text{ cohort-cutoff}_{country}\}}
	\end{aligned}
	$
\par}%Necessary for centering to work

In our analysis, we compare all individuals in the treatment group with all individuals in the control group, irrespective of their educational level. Since all individuals in the control group are older than those in the treatment group at the time of data collection, they have had more time to complete any advanced education. To address this, we restrict our sample to individuals older than 23, ensuring that individuals in the treatment group have likely finished their advanced education. However, due to the timing of the reforms, this restriction affects only a few individuals. A robustness check, excluding this age restriction, is provided in the Appendix (Tables \ref{tab:all_age_educ} to \ref{tab:all_age_prefs}). Additionally, we create 3, 5, and 10-years samples. The \textit{x}-sample is restricted to individuals born within \textit{x} years before or after the cohort cutoff---the cutoff, which separates affected and unaffected individuals. This leaves us with 5,252 observations in the 10-year sample, 2,983 observations in the 5-year sample, and 1,815 observations in the 3-year sample (see Tables \ref{tab:nobs_5y} to \ref{tab:nobs_10y} for sample sizes per reform).\footnote{For data preparation and analysis we use the template for reproducible research in economics by \citet{von_gaudecker_templates_2019}.}

\begin{table}[htbp]
	\caption{Descriptive statistics}
	\label{tab:descriptives}
	\centering
	\begin{threeparttable}
		\input{../bld/python/tables/descriptive_statistics.tex}
		\begin{tablenotes}
			\footnotesize
			\item \textit{Notes:} The sample contains individuals who are born within five years before or after a cohort cutoff and are older than 23. Standard errors are clustered on the reform $\times$ birth-year level. \textit{Data: STEP}
		\end{tablenotes}
	\end{threeparttable}
\end{table}

Table \ref{tab:descriptives} shows descriptive statistics for individuals in the 5-year sample, comparing those unaffected (control group) and affected (treatment group) by the reform. Average years of education per reform are displayed in Table \ref{tab:years_educ_per_reform} in the Appendix. Since individuals in the treatment group are, by design, younger, we will account for this in our empirical approach. On average, there is no significant difference in years of education between the treatment and control groups. As we will discuss in the results section, this lack of differences is due to cohort trends and the fact that the first four cohorts in Vietnam were only partially treated. After taking this into account, the treated individuals, on average, have significantly more years of education. Additionally, parental involvement in education is higher in the treatment group, with no significant differences observed in other characteristics. Notably, one quarter of children worked before the age of 15, and this proportion remains unchanged by the introduction of compulsory schooling reforms.



\section{Empirical strategy} \label{sec:emp_strategy}

\subsection{Research design} \label{subsec:conc_frame}
We hypothesize that reforms expanding mandatory education affect non-cognitive skills and, hence, we expect different levels of skills for treated individuals and untreated individuals.

The effect of compulsory schooling reforms on skills could, in principle, be identified from a simple comparison of the average skill level of the first cohort affected by the reform, which we refer to as the pivotal cohort, with the youngest cohort  not affected by the reform, assuming both cohorts were otherwise exposed to the same conditions influencing skill formation (e.g., income, institutions, nutrition, and education quality).
While this assumption may approximately hold for two consecutive cohorts, it becomes less plausible for cohorts born several years apart due to time trends in factors that affect skill formation. Given that cohort sizes in the dataset are too small to reliably estimate the average treatment effect by comparing only the two adjacent cohorts, we extend the analysis to include multiple cohorts in both treatment and control groups. This necessitates controlling for cohort trends to account for evolving conditions that influence skill accumulation. As we explain below, we address this by specifying flexible cohort trends. Note that by controlling for cohort trends, we also control for age trends in skills as age and cohort are collinear in the absence of panel data.\footnote{Our data is plotted in Figure \ref{fig:RD-style} in addition to linear cohort trends.}


Our primary interest lies in understanding the effect of the reforms on skills. As a direct consequence of the reforms, years of education are expected to increase for individuals who, in the absence of the reform, would have left school before meeting the new compulsory education  requirements. Some students might even choose to continue their education beyond the mandatory years, inspired or motivated by their experiences during compulsory schooling. At the same time, spillover effects may arise for children who would have surpassed the new mandatory years of education even in absence of the reform---for example, if they aim to distinguish themselves from peers who only complete the increased mandatory education. Thus, the reforms likely influence educational attainment for multiple types of students.


In section \ref{sec:mechanisms}, we explore additional pathways through which educational expansions might affect non-cognitive skills, beyond changes in years of education. These potential mechanisms include improvements in educational quality, shifts in the perceived importance of education, and changes in ability mixing within schools. However, our empirical strategy is designed to capture the overall impact of increased mandatory education, rather than isolating the effects of individual mechanisms.



\subsection{Main empirical design} \label{subsec:emp_main}

Our identification strategy is to use within-country variation in mandatory years of education. From an individual's perspective, the timing of the reform can be considered as-good-as-random, providing a quasi-experimental design.

Individuals are classified into a control and a treatment group:

{\centering
	$ \displaystyle
	\begin{aligned}
		D_{i, country} = \mathds{1}_{\{\text{birth-year}_{i, country} \text{ } \geq \text{ cohort-cutoff}_{country}\}}
	\end{aligned}
	$
\par}%Necessary for centering to work


Our empirical design exploits a discontinuous change in the outcome variable when the year of birth reaches the cohort cutoff. Under certain conditions, changes in the observed outcome at this cutoff can be attributed to the change in treatment status. The key identifying assumption is that potential outcomes evolve smoothly around the cutoff. If a jump in potential outcomes occurred independently of the reform at the threshold, it would be impossible to disentangle the causal effect of treatment from pre-existing changes. This assumption can be interpreted as the impossibility of manipulating treatment assignment. It is important to note that our design only allows causal inference at or near the cutoff value, as discussed in section \ref{subsec:conc_frame}.


{\centering
	$ \displaystyle
	\begin{aligned}
		\mathbb{E}[Skill^d|X=x] \text{ continuous in } x \text{ around cutoff } c \text{ for } d \in {0,1}
	\end{aligned}
	$
\par}%Necessary for centering to work


Can treatment be considered exogenous in our setting? While treatment was not strictly random---for instance, if some parents deliberately tried to postpone or advance the birth of their children---the implementation of a reform typically follows a complex political process that unfolds over several years. This makes the precise timing of a reform effectively random for a given birth cohort. Therefore, from an individual's perspective, treatment assignment can be considered exogenous.

Ideally, we would compare individuals just at the cutoff---the youngest untreated and the oldest treated cohort---as they are likely to be most similar in other observed or unobserved characteristics. However, given the nature of our data, such a narrow sample would be too small for reliable inference. Therefore, we expand the sample to include a broader range of birth cohorts and estimate cohort trends in skills. Since skills tend to be relatively stable over a limited age range, we are confident that this approach provides a valid estimate of the treatment effect.

Our empirical design resembles a regression discontinuity design (RDD) \parencite{thistlethwaite_regression-discontinuity_1960} in that it exploits a discontinuous jump in the outcome variable at a cutoff point,  with the year of birth serving as the forcing variable. This approach is widely used in the education literature. However, our empirical design is not identical to an RDD, as we estimate cohort trends for each reform separately and aggregate the effects to obtain an overall reform effect.\footnote{
	An alternative to our main empirical approach would be to use an instrumental variable (IV) design, where an indicator for being affected by a reform serves as the instrument for years of education. The aim of this approach would be to estimate the effect of one additional year of education, rather than the overall reform effect. However, we prefer the previously outlined empirical strategy over an IV approach, due to concerns raised in the recent literature regarding IV assumptions and the questionable exclusion restriction. In our context, with cross-sectional data from multiple countries and reforms, the IV assumptions could only be valid conditional on cohort and reform fixed effects. However, \citet{blandhol_when_2022} showed that the IV approach is problematic in the presence of covariates and heterogeneous treatment effects. They found that a 2SLS estimate may not even be a positively weighted average of causal effects in this case unless based on a fully saturated model. A fully saturated model in our case would risk sever severe overfitting. Moreover, the exclusion restriction assumes that reforms increasing mandatory education affect skills skills solely through years of education, but as discussed in subsection \ref{subsec:conc_frame} and section \ref{sec:mechanisms}, there are reasons to question this assumption. Therefore, we opt for our previously described approach, as it reliably provides an overall average treatment effect of increasing years of compulsory education.
}

We estimate variants of the following baseline regression model, where $\textit{skill}_{icr}$ represents a non-cognitive skill of individual $i$ from birth cohort $c$, who belongs to either the control or treatment group of reform $r$.


\begin{align} \label{eq:rdd}
	\begin{split}
		\text{Skill}_{icr} = & \beta_1 \text{Treated}_{cr} + \beta_2 \text{Partially-treated}_{cr} + \sum_{r} \gamma_{1r} \text{Reform}_{r} \\
		& + f(\text{Cohort}_{icr} \times \text{Reform}_r) \\
		& + \text{Treated}_{cr} \times f(\text{Cohort}_{icr} \times \text{Reform}_r) \\
		& + \varepsilon_{icr}
	\end{split}
\end{align}


To account for systematic differences in skill levels across countries and calendar time, we include reform fixed effects. Moreover, to account for systematic differences in cohort trends across countries, we include interactions between the cohort and reform fixed effects. Additionally, we allow for distinct trends before and after a reform, as education may affect not only skill levels but also the rate of skill development. As described in section \ref{sec:vietnam}, the implementation of the compulsory schooling reform in Vietnam was gradual rather than immediate. Following \citet{cornelissen_multigenerational_2022}, we classify the first four treated cohorts in Vietnam as partially treated.

To ensure balance across treatment and control group in observed characteristics, we estimate equation (\ref{eq:rdd}) with these characteristics as the dependent variable. Table \ref{tab:balanced} shows that the sample is balanced across most observed characteristics, except for the number of siblings at age twelve. Therefore, we control for the number of siblings at age twelve in our analysis.


\begin{table}[htbp]
	\caption{Balanced sample}
	\label{tab:balanced}
	\centering
	\begin{threeparttable}
			\input{../bld/python/tables/balanced_sample.tex}
		\begin{tablenotes}
			\footnotesize
			\item \textit{Notes:} Individuals born within five years before or after a cohort cutoff and older than 23. Standard errors are clustered on the reform $\times$ birth-year level. \textit{Data: STEP}
		\end{tablenotes}
	\end{threeparttable}
\end{table}




To sum up, as the validity of our estimate for the average treatment effect depends on the adequate representation of $\mathbb{E}[Skill|X]$ around the cutoff, our baseline specification identifies the causal effect of expanding compulsory education if (i) cohort trends are well approximated and (ii) no other event affects the jump at the cutoff.


\subsection{Expected effects of compulsory schooling on non-cognitive skills} \label{subsec:exp_effects}

Before presenting the results of our main analysis, we briefly outline the expected findings based on existing literature and the definitions of the skills in question. Despite the limited number of causal studies discussed in the introduction, expectations can be formed from correlations between educational attainment and non-cognitive skills identified in previous research. These correlations may reflect causal effects, consequences of educational attainment, or spurious relationships.

Based on the summary in \citet{almlund_chapter_2011}, we might expect education to increase openness to experience, conscientiousness, and emotional stability, while its impact on extraversion and agreeableness may be negligible. We anticipate that decision-making patterns would improve with more compulsory education, as education enhances cognitive skills that facilitate the consideration of alternative solutions and the evaluation of future consequences.

A positive effect on grit is also plausible, as suggested by \citet{alan_ever_2019}. Hostile attribution bias, which may be inversely related to trust, could decline with increased education, given the positive correlation between trust and education found in previous research \parencite{oreopoulos_priceless_2011, charron_does_2016}. Moreover, \citet{yang_does_2019} and \citet{kan_educated_2021} report a (weak) causal link between education and trust, supporting the expectation of a negative effect of compulsory schooling reforms on hostile attribution bias.

For economic preferences, previous causal evidence is mixed \parencite{dohmen_effect_2022, tawiah_does_2022, jung_does_2015, jung_does_2021}. However, \citet{alan_fostering_2018} show that a school-based curriculum intervention increases patience, and \citet{sutter_impatience_2013} find that patient children exhibit better school conduct, even though they do not directly study the link with educational attainment. These findings suggest a potential positive effect of the reforms on patience. In contrast, we do not have a clear expectation regarding the impact on willingness to take risks.


\section{Results} \label{sec:results}
\subsection{Main results} \label{subsec:res_main}

The results from our empirical approach, as described in Section \ref{subsec:emp_main}, are presented in Tables \ref{tab:res_years_educ} to \ref{tab:res_prefs}. The dependent variables are grouped into three categories: education, standardized non-cognitive skills, and binary non-cognitive skills. We use three alternative time windows around the pivotal cohorts to test the robustness of our findings.

To guide the choice of an optimal window size, we rely on the Python package \textit{rdrobust} by \citet{calonico_rdrobust_2023}, which provides the Mean Squared Error (MSE)-optimal bandwidth. For most outcomes, the optimal bandwidth is approximately 3 years (see Table \ref{tab:bandwidth} in the Appendix). This bandwidth selection accounts for uniform weights, clustered standard errors, and the inclusion of covariates. However, the package does not provide the option to estimate separate trends for each reform. As the reforms occurred at different points in time, individuals in our sample were surveyed at varying ages. As a result, estimating separate cohort trends for each reform is essential. Because estimating multiple cohort trends requires a larger sample size, we prioritize a larger window size. To minimize potential confounding factors, we base our main interpretation on the smallest possible bandwidth that remains larger than the MSE-optimal bandwidth. Thus, we choose the 5-year window as our preferred window size, balancing sample size and distance to the reform year best. Nevertheless, we present results for the 3-, 5-, and 10-year samples to assess the robustness of our findings.

In addition to varying the bandwidth, we present results from different model specifications to, further, assess the robustness of our findings. Columns (1), (3), and (5) report estimates from a linear model with a linear cohort trend. Columns (2), (4), and (6) show estimates from a quadratic model, while column (7) presents estimates from a cubic model. Given the discrete nature of the cohort variable and the relatively short time periods around the reforms, we prioritize model simplicity to mitigate the risk of overfitting cohort trends. For this reason, we consider the linear model the most appropriate specification, particularly for the preferred 5-year sample. Therefore, the linear model with the 5-years sample is our preferred specification.



\begin{table}[htbp]
	\caption{Years of education}
	\label{tab:res_years_educ}
	\centering
	\begin{threeparttable}
		\footnotesize
		\input{../bld/python/tables/with_partially_treated/common_trend/results_with_partially_treated_years_educ_tabular.tex}
		\begin{tablenotes}
			\footnotesize
			\item \textit{Notes:} Estimated average treatment effects of compulsory schooling reforms. The sample is restricted to individuals who are older than 23 and to the following reforms: Bolivia 1994, Colombia 1991, Ghana 1961, Ghana 1987, Vietnam 1991. The \textit{linear/quadratic/cubic} model consists of reform specific \textit{linear/quadratic/cubic} cohort trends. Standard errors are cluster robust on reform $\times$ birth-year level. \textit{Data: STEP} $^{***} p < 0.01, ^{**} p < 0.05, ^{*} p < 0.1$
		\end{tablenotes}
	\end{threeparttable}
\end{table}


As expected, we find a positive effect of compulsory schooling reforms on years of education (Table \ref{tab:res_years_educ}). The estimate from our preferred model indicates that the reforms increased years of education by 0.95 years on average.  Given that the average increase in mandatory education across the reforms under study is 4 years, this result suggests a meaningful impact. The estimates remain consistent across most specifications, with the exception of the quadratic model in the 10-year sample, where the effect size is slightly smaller. Notably, the coefficients in the second row correspond to the first four treated cohorts in Vietnam. In line with expectations, we observe no significant increase in years of education for these cohorts, as the reform effect at the cutoff is effectively zero (-0.01) in our preferred specification, reflecting the gradual implementation of the reform.

The primary focus of our analysis is the effect of compulsory schooling reforms non-cognitive skills. Our estimates show that while increased years of compulsory education do not affect all non-cognitive skills, they do influence most of them. The results from our preferred specification using the 5-year sample are summarized in Figure \ref{fig:main_results}.\footnote{Estimates for each reform separately are provided in Table \ref{tab:single_reforms}.} More detailed results can be found in Tables \ref{tab:res_ncogn} and \ref{tab:res_prefs}.

Although point estimates and significance levels vary depending on the sample size and specification, except for the positive estimate for patience in column (4), significant estimates consistently show the same sign across different models. For the reasons outlined above, we base our main interpretation on the results from the linear model and the 5-years sample.


\begin{figure}[htbp]
	\centering
	\caption{Effect of increasing mandatory education on non-cognitive skills}
	\label{fig:main_results}
	\includegraphics[width = 0.95\columnwidth, trim = {1.5cm 2cm 0.5cm 5.1cm}, clip]{../bld/python/figures/results_figure_two_xaxis.png}
	\caption*{\footnotesize \textit{Notes:} Estimated average treatment effects of compulsory schooling reforms. Point estimates and 95\%-confidence intervals are displayed. The sample is restricted to individuals who are older than 23 and to the following reforms: Bolivia 1994, Colombia 1991, Ghana 1961, Ghana 1987, Vietnam 1991. The estimates stem from the 5 years sample and the \textit{linear model} which has linear cohort trends. Standard errors are cluster robust on reform $\times$ birth-year level. \textit{Data: STEP}}
\end{figure}

\begin{table}[htbp]
	\caption{Personality traits and behaviors}
	\label{tab:res_ncogn}
	\centering
	\begin{threeparttable}
		\footnotesize
		\input{../bld/python/tables/with_partially_treated/common_trend/results_with_partially_treated_ncogn_skills_tabular.tex}
		\begin{tablenotes}
			\footnotesize
			\item \textit{Notes:} Estimated average treatment effects of compulsory schooling reforms. Measures are standardized. The sample is restricted to individuals who are older than 23 and to the following reforms: Bolivia 1994, Colombia 1991, Ghana 1961, Ghana 1987, Vietnam 1991. The \textit{linear/quadratic/cubic} model consists of reform specific \textit{linear/quadratic/cubic} cohort trends. Standard errors are cluster robust on reform $\times$ birth-year level. \textit{Data: STEP} $^{***} p < 0.01, ^{**} p < 0.05, ^{*} p < 0.1$
		\end{tablenotes}
	\end{threeparttable}
\end{table}

\begin{table}[htbp]
	\caption{Preferences}
	\label{tab:res_prefs}
	\centering
	\begin{threeparttable}
		\footnotesize
		\input{../bld/python/tables/with_partially_treated/common_trend/results_with_partially_treated_preferences_binary_tabular.tex}
		\begin{tablenotes}
			\footnotesize
			\item \textit{Notes:} Estimated average treatment effects of compulsory schooling reforms. Measures are binary. The sample is restricted to individuals who are older than 23 and to the following reforms: Bolivia 1994, Colombia 1991, Ghana 1961, Ghana 1987, Vietnam 1991. The \textit{linear/quadratic/cubic} model consists of reform specific \textit{linear/quadratic/cubic} cohort trends. Standard errors are cluster robust on reform $\times$ birth-year level. \textit{Data: STEP} $^{***} p < 0.01, ^{**} p < 0.05, ^{*} p < 0.1$
		\end{tablenotes}
	\end{threeparttable}
\end{table}


We find that individuals affected by the reform experience a reduction in emotional stability by 0.23 standard deviations, while openness to experience increases by 0.14 standard deviations. The quadratic model suggests an increase in conscientiousness by 0.28 standard deviations and in agreeableness by 0.37 standard deviations, although these estimates are not statistically significant in our preferred linear model.
The reforms also improve decision-making patterns by 0.10 standard deviations, reduce grit by 0.14 standard deviations, and lower hostile attribution bias by 0.21 standard deviations. Regarding economic preferences, we find a negative effect of the reforms on both patience and willingness to take risks. Being affected by a reform reduces the probability to be patient by 6\% and the probability to be willing to take risk by 5\%.

The magnitude of the effects we find is in line with those reported in other studies. For example, \citet{alan_ever_2019} show that a curriculum intervention increased students' self-reported grit by 0.29 to 0.35 standard deviations, depending on the sample. Similarly, \citet{dasgupta_effects_2022} find that extraversion and conscientiousness among men marginally admitted to a selective college decrease by 0.48 and 0.56 standard deviations, respectively. Another relevant example is the study by \citet{dahmann_cross-fertilizing_2018}, which finds that Germany's G8 school reform reduces emotional stability by approximately one-third of a standard deviation. In comparison, our estimates---ranging from 0.1 to 0.23 standard deviations in absolute terms under our preferred specification---are of a similar order of magnitude, leaning towards the more modest end of the spectrum.

In summary, the reforms expanding compulsory education do not have a significant impact on agreeableness, conscientiousness, or extraversion according to our preferred specification and sample. However, the reforms lead to a decrease in emotional stability, grit, hostile attribution bias, patience, and willingness to take risks while increasing openness to experience and decision-making patterns, i.e., alternative solution and consequential thinking. The most consistent findings across samples and specifications are the increases in openness to experience and decision-making patterns, as well as the decrease in hostile attribution bias---effects that align with expectations from the literature. Additionally, we can confirm that the reforms lead to an increase in years of education.

\subsection{Robustness of results}
To assess the robustness of our results, we conduct several additional analyses, which are described in this section. Before presenting the results of our robustness checks, we first run a placebo test. Using our preferred specification and sample, we shift all pivotal cohorts seven years forward (Table \ref{tab:placebo}). This seven-year shift ensures that we maintain a sufficient number of observations and avoid overlap with any other compulsory schooling reforms during the period considered. As expected, the significant effects identified in our main analysis disappear in the placebo specifications.

As a first robustness check, we classify individuals born one year before the pivotal cohort as partially treated. This adjustment addresses the concern that using calendar-year cutoffs to define treatment and control groups might result in partially treated cohorts within the control group. This could occur if some children enter primary school late or repeat a grade.\footnote{
	The data set would allow us to use birth month instead of birth year to determine treatment based on school entry cutoff. However, we decided against this for several reasons: (1) Cutoff dates at the time when reforms were implemented are not known, and it is unclear whether they are the same as today. (2) Even if historical cutoff dates matched current ones, they may not have been strictly enforced. (3) Some individuals did not report their birth month, likely because they did not know their exact birth date. We explored how our results change when applying the best available guesses for cutoff dates. With these dates, our main findings remain largely robust.
}
It is important to note that excluding cohorts hinders the comparability of treated and untreated individuals. Column (1) in Tables \ref{tab:robust_ncogn} and \ref{tab:robust_prefs} show the results when we classify the last ``unaffected'' cohort as partially treated (and treated), similar to our approach for the first four cohorts in Vietnam. With this approach, we observe similar effects on emotional stability, grit, hostile attribution bias, and patience. The estimated effects on openness to experience, decision-making patterns, and willingness to take risks are not significantly different from zero, though all estimates have the same sign.

Second, we use an inflexible model to, further, test the robustness to alternatively modeling cohort trends. The inflexible model imposes the same slope before and after the cutoff. Intuitively, this restriction implies that the additional compulsory education only affects skills' levels but not the rate of skill development. Estimates from this model are provided in column (2) of Tables \ref{tab:robust_ncogn} and \ref{tab:robust_prefs}. The significance levels differ for some non-cognitive skills, but the point estimates are similar to our main results.

Third, we test whether the main results in Figure \ref{fig:main_results} are robust to excluding observations from one reform at a time, to check if any effect is primarily driven by a specific reform. Some findings---such as those related to emotional stability and hostile attribution bias---are more robust than other findings---such as the results for willingness to take risks---when a single reform is excluded. However, all findings remain qualitatively robust (see Figure \ref{fig:leave_one_out}).


Finally, we account for multiple hypothesis testing by applying a sharpened False Discovery Rate (FDR) control, using the Stata implementation by \citet{anderson_multiple_2008}, which is based on the algorithm of \citet{benjamini_adaptive_2006}. The original p-values, along with the q-values obtained from the sharpened FDR control, are presented in Table \ref{tab:qvalues}. Overall, the robustness checks support the conclusion that compulsory education has an effect on non-cognitive skills.




%%%%%%%%%%%%%%%%%%%%%%%%%%%%%%%%%%%%%%%%%%%%%%%%%%%%%%%%%%%

\subsection{Returns to skills and labor market outcomes}

In this section we explore whether the compulsory schooling reforms improved skills that are valued in the labor market. A priori, it is not clear whether an increase or decrease in a particular non-cognitive skill will be rewarded in the labor market. To assess this, we estimate wage returns to skills to gauge whether these were affected in a way that could increase productivity in the labor market. For this purpose, we estimate augmented Mincer wage regressions with the full sample of working individuals in Bolivia, Colombia, Ghana, and Vietnam. Furthermore, we investigate if labor market outcomes differ across control and treatment groups.

The Mincer wage regression estimates are shown in Table \ref{tab:wage_returns}. In line with previous studies, we find a gender wage gap of about 20\%, a concave effect of age and a positive effect of education on wages. Importantly, the findings from the augmented Mincer wage regression in column (6) show that an increase in openness to experience and a decrease in hostile attribution bias are positively related to wages, while a reduction in willingness to take risks is negatively related to wages. For the other non-cognitive skills that are affected by the expansion of mandatory education---emotional stability, decision-making patterns, grit, and patience---we find neither a positive nor a negative relationship with wages. Extraversion is positively related to wages but not affected by the reforms. Notably, all self-reported cognitive skills (reading, writing, and numeracy) are positively related to wages.


Furthermore, these descriptive results suggest that compulsory schooling reforms affect wages by increasing years of education (column 1). This is partly driven by changes in cognitive skills, as evidenced by the lower correlation between years of education and wages when cognitive skills are controlled for (see columns (2) and (6)). However, changes in non-cognitive skills do not appear to mediate the relationship between years of education and wages, since controlling for non-cognitive skills does not alter the coefficient of years of education (see columns (3) to (5)).

To estimate the log wage change for an average individual affected by a reform, we use model (6). Specifically, we multiply the wage returns from model (6) that are significantly different from zero by the significant changes in skills and years of education, which are estimated using our preferred specification with the 5-year sample (column (1) in Tables \ref{tab:res_ncogn} and \ref{tab:res_prefs}). The resulting predicted log wage change is 0.091.
%\input{../bld/python/other/predicted_wage_change_all_skills.tex}
The log wage change due to affected non-cognitive skills is
%\input{../bld/python/other/predicted_wage_change_non-cognitive_skills.tex}
0.036, with the main driver being reduced hostile attribution bias. This suggests that approximately 40\% of the overall predicted log wage change can be attributed to the changes in non-cognitive skills.

In addition to predicting wage changes based on changes in skills and years of education, we estimate the reform effects on labor market outcomes by running the same regression as in equation \ref{eq:rdd}, but with wages, an indicator for currently working, and an indicator for being a wage worker versus self- or family-employed as dependent variables. It is important to recall that the individuals in the treatment group are younger than those in the control group. Thus, even though we control for age, the labor market outcomes for treated individuals might differ due to unobserved factors that are related to age but not perfectly explained by it. Since personality traits and preferences tend to remain stable within the limited age range considered, cohort trends effectively capture age-related effects in labor market outcomes.

The results from this exercise are presented in Table \ref{tab:lm_outcomes}. In column (1) of Table \ref{tab:lm_outcomes}---corresponding to our preferred specification, we find no significant differences in hourly earnings, the probability to be currently working, or the likelihood of being a wage worker versus self- or family-employed between the control and treatment groups. However, using the quadratic model or the 3-years sample (Columns (2) to (4)), we find a significant increase in wages for treated individuals.
Additionally, we find no systematic differences in the occupations of individuals in the treatment group compared to those in the control group (see Figure \ref{fig:occupations}). Any small differences in occupation groups 1 to 4 can likely be attributed to the younger age of treated individuals.


\section{Discussion} \label{sec:discussion}
Our results show that reforms increasing compulsory education significantly affect personality traits, economic preferences, decision-making patterns, grit, and hostile attribution bias of treated individuals. The positive effect on openness to experience and the negative effect on hostile attribution bias align with our expectations based on correlations reported in the literature (see Section \ref{subsec:exp_effects}).

In contrast, the negative effect on emotional stability contrasts with our expectations. One possible explanation is that education, by broadening individuals' knowledge of the world and their own limitations, may increase their tendency to worry, thereby reducing emotional stability. The negative impact on patience is also rather surprising given prior research, though evidence on this relationship remains limited for low- and middle-income countries. For instance, while \citet{dohmen_effect_2022} overall find a positive effect of compulsory schooling reforms on education, they report no significant impact on patience in low- and middle-income countries. The relatively low life expectancy in low- and middle-income countries might help explain this result. Research indicates that life expectancy increases both patience \parencite{becker_endogenous_1997, falk_longevity_2019} and educational attainment \parencite{ben-porath_production_1970, cervellati_life_2013}. In settings with shorter life expectancy and potentially lower educational quality, being compelled to stay in school longer---rather than choosing one's optimal level of education---may lead some individuals to perceive additional years as lost time, ultimately reducing their patience. This interpretation suggests that the effect of compulsory education on patience may be context-dependent.
The negative effect on grit is also unexpected. However, given the reduction in patience, a decline in grit---defined as passion and perseverance for long term goals---seems not surprising. Finally, our analyses show that expanding compulsory education decreases willingness to take risk, consistent with the findings by \citet{jung_does_2015} for Great-Britain.

We acknowledge that compulsory schooling reforms likely have heterogeneous effects on students. In particular, we expect these reforms to primarily, though not exclusively, affect individuals with lower levels of education and potentially disadvantaged socio-economic backgrounds. Therefore, our estimates may not be fully generalizable to the entire population but rather to individuals whose characteristics are similar to those increasing their years of education due to the reform. From a policy perspective, however, this subgroup is highly relevant. Beyond the increase in years of education, other potential mechanisms may contribute to our findings. We discuss these mechanisms below, followed by a consideration of data limitations.


\paragraph{Potential mechanisms} \label{sec:mechanisms} \mbox{}\\
Non-cognitive skills could be affected by reforms extending mandatory education through mechanisms beyond the direct increase in years of education. As mentioned in section \ref{subsec:conc_frame}, we explore these other potential mechanisms here.

First, the reform-induced increase in student numbers might strain educational resources, leading to overcrowded classrooms or teacher shortages, and thereby causing a decline in educational quality. Such a decline could adversely affect the non-cognitive skills of students who would have attended school even in the absence of the reforms.\footnote{
	To assess whether  our findings are driven by a decline in educational quality for more educated individuals, we split the sample based on years of education. The sample of less educated individuals is smaller. For both groups, we find no significant effect on agreeableness, extraversion, conscientiousness, or grit. However, emotional stability declines significantly in both groups, with a stronger negative effect for less educated individuals. Openness to experience increases significantly in both groups. Patience decreases similarly across groups, but the effect is only significant for more educated individuals. Hostile attribution bias decreases significantly for more educated individuals, while willingness to take risks declines significantly for less educated individuals. Finally, decision-making patterns improve significantly for more educated individuals.
}
Therefore, it is crucial to determine whether the government anticipated the challenges posed by the reform and responded by increasing the number of school buildings and teachers accordingly. Our data does not include direct measures of school quality, so we rely on information from relevant research papers and reports on the reforms. For four out of the five reforms studied in this article, we found pertinent information. In Ghana, the ``Accelerated Development Plans'' (ADP) for education, launched in 1951, ten years before the compulsory schooling reform \parencite{kadingdi_policy_2006}, lead to a rapid and sustained growth in schools and the number of teachers. Additionally, new teacher training colleges opened in 1961. However, \citet{kadingdi_policy_2006} notes that ``even though school enrollments increased following the 1961 Education Act, the quality of teaching and learning appears to have remained the same.'' In 1987, Ghanaian government received World Bank credits to finance the expansion of mandatory education, including new infrastructure \parencite{kadingdi_policy_2006}. In Colombia, an ambitious school construction initiative accompanied the 1991 education reform \parencite{urbina_mass_2022}, while the 1991 reform in Vietnam saw increased investments aimed at building new schools, training additional teachers, and providing financial support to disadvantaged students \parencite{cornelissen_multigenerational_2022, nguyen_trends_2004}. These examples suggest that maintaining educational quality was a key priority for the governments overseeing these reforms.


Second, increasing the years of mandatory education might signal the importance of education to parents or students. A student aware of the government's emphasis on education might feel more motivated to study, potentially enhancing educational outcomes. On the other hand, if the student distrusts the government, they might resist this directive, making the impact of this channel on student learning ambiguous. Parents who perceive education as more important after the reform might invest more in their children's education and possibly encourage them to pursue higher education. Our data set includes a variable measuring parental investment in education, specifically whether parents actively monitored their children's exam results or grades. As shown in Table \ref{tab:descriptives}, parental investment is significantly higher among treated individuals. However, after controlling for birth year and reform fixed-effects, this difference is no longer statistically significant (see Table \ref{tab:balanced}).

Third, mandatory education leads to a more diverse mix of students in school. For instance, children from a low-socioeconomic background, who might not attend school in absence of mandatory education, now do so. As a result, students affected by the reform are exposed to peers with different backgrounds and abilities. \citet{ahn_long-term_2021} find that ability mixing in education reduces individuals' agreeableness and conscientiousness. However, we observe no effect on these traits in our data. This could indicate that ability mixing was not particularly prevalent as a result of the reforms, or that any potential positive impact of education on conscientiousness and agreeableness was offset by the effects of ability mixing.


\paragraph{Limitations} \label{sec:limitations} \mbox{}\\
Our study has several limitations, primarily stemming from data-related challenges.

First, measuring non-cognitive skills is inherently challenging and prone to measurement error, especially when using a short version of a personality inventory with only three items per trait. While such short inventories are valuable in time-limited surveys, having fewer items increases measurement error \parencite{dohmen_accounting_2024}. Similarly, in the STEP data, skills are measured with just two to four items, which likely increases our standard errors due to measurement error in the dependent variable. Additionally, measuring skills in low- and middle-income countries poses extra challenges \parencite{bauer_using_2020, laajaj_challenges_2019}. \citet{laajaj_challenges_2019} found that a clear Big Five factor structure is often lacking in data from non-WEIRD populations, suggesting that survey questions may capture multiple traits instead of specific ones. Despite these challenges, our findings remain robust when replacing specific survey items as shown in column (3) of Table \ref{tab:robust_ncogn}.\footnote{
	\citet{laajaj_challenges_2019} propose alternative measures for the Big Five personality traits based on the same data as we are using. Using a psychometric approach, the authors show that---except for emotional stability---the survey questions do not always measure the intended trait. Instead, they may capture attributes of a different trait. To address this, we leverage \citeauthor{laajaj_challenges_2019}'s \parencite*{laajaj_challenges_2019} findings in Table 1 to replace survey items that capture attributes of other traits with items intended to measure other traits but that also capture attributes of the given trait. The results of this adjustment are presented in column (3) of Table \ref{tab:robust_ncogn}. Our main findings remain robust to replacing these survey items. Additionally, we find a positive effect on agreeableness of increasing years of compulsory education.
} \citet{laajaj_challenges_2019} also note that the lack of a clear Big Five factor structure in survey data from non-WEIRD populations is not due to education levels, but rather the influence of social desirability in face-to-face interviews.


Another limitation is that we observe individuals' skills only once, typically when they are in their 30s. While our empirical design enables us to attribute the estimated differences between treatment and control groups to the increase in compulsory education years, it does not allow us to disentangle the channels affecting skills during education from those influencing skills during labor market participation. For instance, an individual who obtained more education due to the reform might also work in a different occupation. Likewise, the larger number of educated young individuals could alter labor market competition. As a result, rather than pinpointing the specific channels through which mandatory education affects non-cognitive skills, we estimate the equilibrium effect of the reform.

Furthermore, our analysis is limited to four countries surveyed in the STEP program that implemented a compulsory schooling reform affecting individuals born between 1947 and 1989.\footnote{There are some more countries contained in the STEP program that also experienced compulsory schooling reforms, but these reforms took place in the late 1990s or early 2000s.} While we do observe some individuals born after 1989, many of them likely had not completed their education at the time of data collection, making them incomparable to older individuals who had already finished their education.

When interpreting the results, it is important to keep in mind that we estimate an overall effect across a diverse set of countries and reforms. Although all countries are low- and middle-income countries, they are spread across three different continents. Moreover, the reforms vary in scope, affecting primary and lower-secondary education, with the number of mandatory years ranging from three to six years. However, it is reassuring that our results are not driven by any single reform (Figure \ref{fig:leave_one_out}) and that the estimates from single reforms are, even though noisy, predominantly pointing in the same direction (Table \ref{tab:single_reforms}). Due to the limited number of observations per country, we are unable to study individual reforms in isolation.

The limited number of observations presents two additional drawbacks. First, it prevents us from restricting the sample to individuals born just before or just after the first affected cohort, forcing us to rely on estimating cohort trends. Second, it limits our ability to conduct heterogeneity analyses, which could have provided insights into how treatment effects vary across gender or socio-economic background. Despite these limitations, our study is novel in estimating the causal effect of compulsory education reforms on multiple non-cognitive skills.



\section{Conclusion} \label{sec:conclusion}

The primary aim of education is often to enhance cognitive skills that contribute to better life outcomes. However, non-cognitive skills also play a crucial role in shaping labor market success. This paper investigates whether compulsory education reforms can strengthen non-cognitive skills and thereby improve individuals' labor market prospects. Leveraging within-country variation in years of compulsory education across four low- and middle-income countries, we estimate the causal effects of these reforms on non-cognitive skills.

Our findings reveal that extending mandatory education significantly affects a range of non-cognitive skills, including personality traits, economic preferences, decision-making patterns, grit, and hostile attribution bias. Importantly, we find no effect on employment probabilities or transitions from self- or family-employment to wage work, indicating that changes in these labor market outcomes do not drive our results.

To evaluate the impact of changes in non-cognitive skills on earning potential, we estimate augmented Mincer wage regressions. Our results suggest that reform induced changes in non-cognitive skills positively contribute to wages. Accounting for the effects on years of education and cognitive skills, we predict an overall wage increase of 9\% based on the treatment effects of the reforms and Mincer wage regressions.

These findings have several important implications, highlighting that non-cognitive skills---like cognitive skills---can be shaped through mandatory education. Given that these skills are rewarded in the labor market, our results highlight the importance of designing education systems that effectively enhance these skills. Our results suggest that education policies should incorporate strategies to develop personality traits, preferences, foster traits like grit, decision-making patterns, and preferences, while mitigating hostile attribution bias. The relative importance of specific non-cognitive skills may vary across contexts, suggesting that education policies aimed at enhancing productivity and earnings should be tailored to the skill demands of each labor market, especially when comparing high-income and low- to middle-income countries.


We also explore potential mechanisms beyond years of education through which compulsory education reforms might influence non-cognitive skills, such as improvement in education quality, shifts in the perceived value of education, and changes in peer ability composition. While we cannot rule out the influence of other factors, our analysis suggests that these alternative mechanisms are unlikely to be the primary drivers of the observed effects.

However, the returns to non-cognitive skills in the countries we studied differ from those in high-income countries. Additionally, the effect of education on patience in our sample contrasts with findings from developed countries, suggesting that our results may not generalize to high-income settings. Future research should investigate whether education enhances non-cognitive skills in developed countries, especially as the growing impact of artificial intelligence may reduce the demand for certain cognitive skills while amplifying the value of certain non-cognitive skills.


\setstretch{1}
\printbibliography
\setstretch{1.5}


\appendix
\section{Additional data details}
\setcounter{table}{0}
\setcounter{figure}{0}
\renewcommand{\thetable}{\Alph{section}.\arabic{table}}
\renewcommand\thefigure{\Alph{section}.\arabic{figure}}


% CORRELATIONS IN THE LITERATURE AND IN OUR DATA
\begin{table}[htbp]
	\centering
	\caption{Correlations in the literature and in our data}
	\label{tab:correlations}
	\begin{threeparttable}
	\begin{tabular}{llp{2cm}p{2.2cm}}
		\hline%
		\hline%
		Non{-}cognitive skill & Characteristics & Correlation in our data & Sign of correlation in the literature \\%
		\hline%
		Agreeableness & Female & {-}0.07*** & +\tnote{(a)}\\
		& Age & 0.04*** & +\tnote{(b)}\\%
		Conscientiousness & Female & {-}0.10*** & none\tnote{(a)}\\
		& Age & {-}0.03* & +\tnote{(b)}\\%
		Emotional stability & Female & {-}0.24*** & $-$\tnote{(a)}\\
		& Age & 0.11*** & +\tnote{(b)}\\
		& Life satisfaction & 0.04** & +\tnote{(c)}\\
		& Abuse before age 15 & {-}0.10*** & $-$\tnote{(d)}\\%
		Extraversion & Female & 0.00 & inconclusive\tnote{(a)}\\
		& Age & {-}0.11*** & inconclusive\tnote{(b)}\\
		& Life satisfaction & 0.16*** & +\tnote{(e)}\\%
		Openness to experience & Female & {-}0.11*** & inconclusive\tnote{(a)}\\
		& Age & {-}0.07*** & $-$\tnote{(b)}\\%
		Decision{-}making patterns & Female & {-}0.05*** & $-$\tnote{(f)}\\
		& Age & 0.00 & weakly +\tnote{(g)}\\
		& Life satisfaction & 0.02 & +\tnote{(h)}\\%
		Grit & Conscientiousness & 0.40*** & +\tnote{(i)}\\%
		Hostile attribution bias & Abuse before age 15 & 0.18*** & +\tnote{(j)}\\%
		Patience & Life satisfaction & 0.04*** & +\tnote{(k)}\\%
		Willingness to take risk & Female & {-}0.04*** & $-$\tnote{(l)}\\
		& Age & {-}0.08*** & $-$\tnote{(l)}\\%
		\hline%
		\hline%
	\end{tabular}
	\begin{tablenotes}
		\footnotesize
		\item \textit{Notes:} In this table we compare correlations in the STEP data with correlations found in the literature. The correlations in our data are based on the 10 years sample. The articles that we refer to serve as examples, they are not an exhaustive list. The measure for decision-making patterns is compared to the decision-making style \textit{vigilance}, which is the most similar style based on the survey items in \citet{mann_melbourne_1997}. We do not have the characteristics that are typically correlated with grit \parencite[as described in][]{alan_ever_2019} available in our data. Therefore, we provide the correlation with conscientiousness, which is typically strongly positively correlated with grit. \textit{Data: STEP} $^{***} p < 0.01, ^{**} p < 0.05, ^{*} p < 0.1$
		\item[(a)] \citet{costa_jr_gender_2001}
		\item[(b)] \citet{roberts_patterns_2006}
		\item[(c)] \citet{hufer-thamm_link_2021}
		\item[(d)] \citet{boillat_neuroticism_2017}
		\item[(e)] \citet{kim_extraversion_2018} %; weaker relationship in non-North American countries
		\item[(f)] \citet{urieta_decision-making_2021}
		\item[(g)] \citet{brown_decision_2011}
		\item[(h)] \citet{deniz_relationships_2006}
		\item[(i)] \citet{duckworth_grit_2007}
		\item[(j)] \citet{zhu_childhood_2020}
		\item[(k)] \citet{schnitker_examination_2012}
		\item[(l)] \citet{falk_global_2018}

	\end{tablenotes}
\end{threeparttable}

\end{table}


% DESCRIPTIVES
\begin{table}[htbp]
	\caption{Number of observations - 5 years}
	\label{tab:nobs_5y}
	\centering
	\begin{threeparttable}
		\input{../bld/python/tables/obs_per_reform_final_5y.tex}
		\begin{tablenotes}
			\small
			\item \textit{Notes:} The sample contains individuals who are born within five years before or after a cohort cutoff and are older than 23. \textit{Data: STEP}
		\end{tablenotes}
	\end{threeparttable}
\end{table}

\begin{table}[htbp]
	\caption{Number of observations - 3 years}
	\label{tab:nobs_3y}
	\centering
	\begin{threeparttable}
		\input{../bld/python/tables/obs_per_reform_final_3y.tex}
		\begin{tablenotes}
			\small
			\item \textit{Notes:} The sample contains individuals who are born within three years before or after a cohort cutoff and are older than 23. \textit{Data: STEP}
		\end{tablenotes}
	\end{threeparttable}
\end{table}

\begin{table}[htbp]
	\caption{Number of observations - 10 years}
	\label{tab:nobs_10y}
	\centering
	\begin{threeparttable}
		\input{../bld/python/tables/obs_per_reform_final_10y.tex}
		\begin{tablenotes}
			\small
			\item \textit{Notes:} The sample contains individuals who are born within ten years before or after a cohort cutoff and are older than 23. \textit{Data: STEP}
		\end{tablenotes}
	\end{threeparttable}
\end{table}


% DESCRIPTIVES PER REFORM
\begin{table}[htbp]
	\caption{Average years of education per reform}
	\label{tab:years_educ_per_reform}
	\centering
	\begin{threeparttable}
		\input{../bld/python/tables/descriptive_statistics_years_educ.tex}
		\begin{tablenotes}
			\small
			\item \textit{Notes:} Descriptive statistics based on individuals who were born within five years before or after a cohort cutoff and who are older than 23. \textit{Data: STEP}
		\end{tablenotes}
	\end{threeparttable}
\end{table}



% TYPICAL RD-STYLE GRAPH
\begin{figure}
	\centering
	\caption{The effect of increasing mandatory education}
	\label{fig:RD-style}
	\begin{subfigure}{0.45\linewidth}
		\includegraphics[width=\linewidth, trim = {1.2cm 4cm 0.2cm 0.2cm}, clip]{../bld/python/figures/RDD_plot_years_educ.png}
		\caption{Years of education}
	\end{subfigure}

	\begin{subfigure}{0.45\linewidth}
		\includegraphics[width=\linewidth, trim = {1.2cm 4cm 0.2cm 0.2cm}, clip]{../bld/python/figures/RDD_plot_agreeableness_av_s_abcorr.png}
		\caption{Agreeableness}
	\end{subfigure}\hfill
	\begin{subfigure}{0.45\linewidth}
		\includegraphics[width=\linewidth, trim = {1.2cm 4cm 0.2cm 0.2cm}, clip]{../bld/python/figures/RDD_plot_conscientiousness_av_s_abcorr.png}
		\caption{Conscientiousness}
	\end{subfigure}

	\begin{subfigure}{0.45\linewidth}
		\includegraphics[width=\linewidth, trim = {1.2cm 4cm 0.2cm 0.2cm}, clip]{../bld/python/figures/RDD_plot_extraversion_av_s_abcorr.png}
		\caption{Extraversion}
	\end{subfigure}\hfill
	\begin{subfigure}{0.45\linewidth}
		\includegraphics[width=\linewidth, trim = {1.2cm 4cm 0.2cm 0.2cm}, clip]{../bld/python/figures/RDD_plot_openness_av_s_abcorr.png}
		\caption{Openness to experience}
	\end{subfigure}

	\begin{subfigure}{0.45\linewidth}
		\includegraphics[width=\linewidth, trim = {1.2cm 4cm 0.2cm 0.2cm}, clip]{../bld/python/figures/RDD_plot_stability_av_s_abcorr.png}
		\caption{Emotional stability}
	\end{subfigure}\hfill
	\begin{subfigure}{0.45\linewidth}
		\includegraphics[width=\linewidth, trim = {1.2cm 4cm 0.2cm 0.2cm}, clip]{../bld/python/figures/RDD_plot_decision_av_s_abcorr.png}
		\caption{Decision-making patterns}
	\end{subfigure}

	\begin{subfigure}{0.45\linewidth}
		\includegraphics[width=\linewidth, trim = {1.2cm 4cm 0.2cm 0.2cm}, clip]{../bld/python/figures/RDD_plot_grit_av_s_abcorr.png}
		\caption{Grit}
	\end{subfigure}\hfill
	\begin{subfigure}{0.45\linewidth}
		\includegraphics[width=\linewidth, trim = {1.2cm 4cm 0.2cm 0.2cm}, clip]{../bld/python/figures/RDD_plot_hostile_av_s_abcorr.png}
		\caption{Hostile attribution bias}
	\end{subfigure}

	\begin{subfigure}{0.45\linewidth}
		\includegraphics[width=\linewidth, trim = {1.2cm 4cm 0.2cm 0.2cm}, clip]{../bld/python/figures/RDD_plot_patience_binary.png}
		\caption{Patience (binary)}
	\end{subfigure}\hfill
	\begin{subfigure}{0.45\linewidth}
		\includegraphics[width=\linewidth, trim = {1.2cm 4cm 0.2cm 0.2cm}, clip]{../bld/python/figures/RDD_plot_risk_binary.png}
		\caption{Risk willingness (binary)}
	\end{subfigure}

	\caption*{\footnotesize \textit{Notes:} Averages of considered outcomes across year of birth. \textit{Data: STEP}}
\end{figure}


% OPTIMAL BANDWIDTHS
\begin{table}[htbp]
	\caption{MSE-optimal bandwidths by \citet{calonico_rdrobust_2023}}
	\label{tab:bandwidth}
	\centering
	\begin{threeparttable}
		\input{../bld/python/tables/with_partially_treated/results_optimal_bandwidth_CCT_bandwidths_only.tex}
		\begin{tablenotes}
			\footnotesize
			\item \textit{Notes:} The sample is restricted to individuals who are older than 23 and to the following reforms: Bolivia 1994, Colombia 1991, Ghana 1961, Ghana 1987, Vietnam 1991. The estimates stem from the \textit{flexible model} which has linear cohort trends and allows for different slopes around the cutoff. Standard errors are cluster robust on country-reform $\times$ birth-year level. \textit{Data: STEP.} $^{***} p < 0.01, ^{**} p < 0.05, ^{*} p < 0.1$
		\end{tablenotes}
	\end{threeparttable}
\end{table}



%%%%%%%%%%%%%%%%%%%%%%%%%%%%%%%%%%%%%%%%%%%%%%%%%%%%%%%%%%%%%
\newpage
\section{Robustness of main findings}
\setcounter{table}{0}
\setcounter{figure}{0}
\renewcommand{\thetable}{\Alph{section}.\arabic{table}}
\renewcommand\thefigure{\Alph{section}.\arabic{figure}}

% SINGLE REFORMS
\begin{table}[htbp]
	\caption{Analysis with single reforms}
	\label{tab:single_reforms}
	\centering
	\begin{threeparttable}
		\footnotesize
		\input{../bld/python/tables/with_partially_treated/single_reforms/results_single_reforms.tex}
		\begin{tablenotes}
			\footnotesize
			\item \textit{Notes:} The estimates stem from the 5 years sample and the \textit{linear model} which has a linear cohort trend. Standard errors are cluster robust on reform $\times$ birth-year level. \textit{Data: STEP}
		\end{tablenotes}
	\end{threeparttable}
\end{table}


% WITHOUT AGE RESTRICTION
\begin{table}[htbp]
	\caption{No age restriction - Years of education}
	\label{tab:all_age_educ}
	\centering
	\begin{threeparttable}
		\footnotesize
		\input{../bld/python/tables/with_partially_treated/without_restricting_age/results_with_partially_treated_years_educ_tabular.tex}
		\begin{tablenotes}
			\footnotesize
			\item \textit{Notes:} Estimated average treatment effects of compulsory schooling reforms. The sample is restricted to individuals who are older than 23 and to the following reforms: Bolivia 1994, Colombia 1991, Ghana 1961, Ghana 1987, Vietnam 1991. The \textit{linear/quadratic/cubic} model consists of reform specific \textit{linear/quadratic/cubic} cohort trends. Standard errors are cluster robust on reform $\times$ birth-year level. \textit{Data: STEP} $^{***} p < 0.01, ^{**} p < 0.05, ^{*} p < 0.1$
		\end{tablenotes}
	\end{threeparttable}
\end{table}

\begin{table}[htbp]
	\caption{No age restriction - Personality traits and behaviors}
	\label{tab:all_age_ncogn}
	\centering
	\begin{threeparttable}
		\footnotesize
		\input{../bld/python/tables/with_partially_treated/without_restricting_age/results_with_partially_treated_ncogn_skills_tabular.tex}
		\begin{tablenotes}
			\footnotesize
			\item \textit{Notes:} Estimated average treatment effects of compulsory schooling reforms. Measures are standardized. The sample is restricted to individuals who are older than 23 and to the following reforms: Bolivia 1994, Colombia 1991, Ghana 1961, Ghana 1987, Vietnam 1991. The \textit{linear/quadratic/cubic} model consists of reform specific \textit{linear/quadratic/cubic} cohort trends. Standard errors are cluster robust on reform $\times$ birth-year level. \textit{Data: STEP} $^{***} p < 0.01, ^{**} p < 0.05, ^{*} p < 0.1$
		\end{tablenotes}
	\end{threeparttable}
\end{table}

\begin{table}[htbp]
	\caption{No age restriction - Preferences}
	\label{tab:all_age_prefs}
	\centering
	\begin{threeparttable}
		\footnotesize
		\input{../bld/python/tables/with_partially_treated/without_restricting_age/results_with_partially_treated_preferences_binary_tabular.tex}
		\begin{tablenotes}
			\footnotesize
			\item \textit{Notes:} Estimated average treatment effects of compulsory schooling reforms. Measures are binary. The sample is restricted to individuals who are older than 23 and to the following reforms: Bolivia 1994, Colombia 1991, Ghana 1961, Ghana 1987, Vietnam 1991. The \textit{linear/quadratic/cubic} model consists of reform specific \textit{linear/quadratic/cubic} cohort trends. Standard errors are cluster robust on reform $\times$ birth-year level. \textit{Data: STEP} $^{***} p < 0.01, ^{**} p < 0.05, ^{*} p < 0.1$
		\end{tablenotes}
	\end{threeparttable}
\end{table}



% PLACEBO TEST
\begin{table}[htbp]
	\caption{Placebo test}
	\label{tab:placebo}
	\centering
	\begin{threeparttable}
		\input{../bld/python/tables/placebo_test/placebo_test.tex}
		\begin{tablenotes}
			\footnotesize
			\item \textit{Notes:} The sample is restricted to individuals born within 5 years before or after the placebo reform cutoff. Placebo reforms are the true reforms plus 7 years. The estimates stem from the \textit{linear model} which has linear cohort trends. Standard errors are cluster robust on country-placebo-reform $\times$ birth-year level. \textit{Data: STEP.} $^{***} p < 0.01, ^{**} p < 0.05, ^{*} p < 0.1$
		\end{tablenotes}
	\end{threeparttable}
\end{table}


% COMBINED TABLE
\begin{table}[htbp]
	\caption{Robustness checks - Personality traits and behaviors}
	\label{tab:robust_ncogn}
	\centering
	\begin{threeparttable}
		\footnotesize
		\begin{tabular}{lccc}%
    \hline%
    \hline%
    &Pre-piv. as partial&Inflexible trend&Replaced items\\%
    &(1)&(2)&(3)\\%
    \hline%
    Dependent variable&&&\\%
    Agreeableness&0.00&0.07&0.09*\\%
    &(0.11)&(0.10)&(0.05)\\%
    Conscientiousness&{-}0.05&0.06&0.03\\%
    &(0.08)&(0.08)&(0.08)\\%
    Emotional stability&{-}0.22**&{-}0.19*&{-}0.22***\\%
    &(0.11)&(0.10)&(0.08)\\%
    Extraversion&0.03&{-}0.04&0.02\\%
    &(0.13)&(0.13)&(0.11)\\%
    Openness to experience&0.08&0.12&0.11*\\%
    &(0.08)&(0.09)&(0.06)\\%
    Decision{-}making patterns&0.09&0.09&\\%
    &(0.07)&(0.07)&\\%
    Grit&{-}0.19**&{-}0.13*&\\%
    &(0.08)&(0.08)&\\%
    Hostile attribution bias&{-}0.22***&{-}0.17***&\\%
    &(0.07)&(0.06)&\\%
    \hline%
    Observations&2447&2447&2449\\%
    \hline%
    \hline%
    \end{tabular}

		\begin{tablenotes}
			\footnotesize
			\item \textit{Notes:} Estimated average treatment effects of compulsory schooling reforms. Measures are standardized. The sample is restricted to individuals who are older than 23 and to the following reforms: Bolivia 1994, Colombia 1991, Ghana 1961, Ghana 1987, Vietnam 1991. Column (1) contains estimates from classifying individuals born one year before the pivotal cohort as partially treated and using the \textit{linear} model. Column(2) contains estimates from an \textit{inflexible} model, which consists of reform specific linear cohort trends that are not flexible at the cutoff, i.e. the model imposes the same slope before and after the cutoff. Column(3) contains estimates from replacing survey items based on \citet{laajaj_challenges_2019}. Standard errors are cluster robust on reform $\times$ birth-year level. \textit{Data: STEP} $^{***} p < 0.01, ^{**} p < 0.05, ^{*} p < 0.1$
		\end{tablenotes}
	\end{threeparttable}
\end{table}

\begin{table}[htbp]
	\caption{Robustness checks - Preferences}
	\label{tab:robust_prefs}
	\centering
	\begin{threeparttable}
		\footnotesize
		\begin{tabular}{lcc}%
    \hline%
    \hline%
    &Pre-piv. as partial&Inflexible trend\\%
    &(1)&(2)\\%
    \hline%
    Dependent variable&&\\%
    Patience&{-}0.10***&{-}0.05**\\%
    &(0.04)&(0.02)\\%
    Willingness to take risk&{-}0.04&{-}0.04\\%
    &(0.05)&(0.03)\\%
    \hline%
    Observations&2964&2964\\%
    \hline%
    \hline%
\end{tabular}

		\begin{tablenotes}
			\footnotesize
			\item \textit{Notes:} Estimated average treatment effects of compulsory schooling reforms. Measures are binary. The sample is restricted to individuals who are older than 23 and to the following reforms: Bolivia 1994, Colombia 1991, Ghana 1961, Ghana 1987, Vietnam 1991. Column (1) contains estimates from classifying individuals born one year before the pivotal cohort as partially treated and using the \textit{linear} model. Column(2) contains estimates from an \textit{inflexible} model, which consists of reform specific linear cohort trends that are not flexible at the cutoff, i.e. the model imposes the same slope before and after the cutoff. Standard errors are cluster robust on reform $\times$ birth-year level. \textit{Data: STEP} $^{***} p < 0.01, ^{**} p < 0.05, ^{*} p < 0.1$
		\end{tablenotes}
	\end{threeparttable}
\end{table}



% LEAVE ONE OUT
\begin{figure}
	\centering
	\caption{Main results leaving one reform out}
	\label{fig:leave_one_out}
	\begin{subfigure}{0.45\linewidth}
		\includegraphics[width=\linewidth, trim = {1.5cm 2cm 0.5cm 5.2cm}, clip]{../bld/python/figures/leave_one_out/results_figure_two_xaxis_leave_Ghana1961_out.png}
		\caption{Leaving Ghana 1961 out}
		\label{fig:a}
	\end{subfigure}\hfill
	\begin{subfigure}{0.45\linewidth}
		\includegraphics[width=\linewidth, trim = {1.5cm 2cm 0.5cm 5.2cm}, clip]{../bld/python/figures/leave_one_out/results_figure_two_xaxis_leave_Ghana1987_out.png}
		\caption{Leaving Ghana 1987 out}
		\label{fig:b}
	\end{subfigure}

	\begin{subfigure}{0.45\linewidth}
		\includegraphics[width=\linewidth, trim = {1.5cm 2cm 0.5cm 5.2cm}, clip]{../bld/python/figures/leave_one_out/results_figure_two_xaxis_leave_Colombia1991_out.png}
		\caption{Leaving Colombia 1991 out}
		\label{fig:c}
	\end{subfigure}\hfill
	\begin{subfigure}{0.45\linewidth}
		\includegraphics[width=\linewidth, trim = {1.5cm 2cm 0.5cm 5.2cm}, clip]{../bld/python/figures/leave_one_out/results_figure_two_xaxis_leave_Vietnam1991_out.png}
		\caption{Leaving Vietnam 1991 out}
		\label{fig:d}
	\end{subfigure}

	\begin{subfigure}{0.45\linewidth}
		\includegraphics[width=\linewidth, trim = {1.5cm 2cm 0.5cm 5.2cm}, clip]{../bld/python/figures/leave_one_out/results_figure_two_xaxis_leave_Bolivia1994_out.png}
		\caption{Leaving Bolivia 1994 out}
		\label{fig:e}
	\end{subfigure}
	\caption*{\footnotesize \textit{Notes:} Point estimates and 95\% confidence intervals. The sample is restricted to individuals who are older than 23 and to the following reforms: Bolivia 1994, Colombia 1991, Ghana 1961, Ghana 1987, Vietnam 1991. Observations from one reform each are left out. The estimates stem from the 5 years sample and the \textit{linear model} which has linear cohort trends. Standard errors are cluster robust on reform $\times$ birth-year level. \textit{Data: STEP}}
\end{figure}



\begin{table}[htbp]
	\caption{Robustness to False Discovery Rate control}
	\label{tab:qvalues}
	\centering
	\begin{threeparttable}
		\input{../bld/python/tables/with_partially_treated/results_with_qvalues.tex}
		\begin{tablenotes}
			\footnotesize
			\item \textit{Notes:} Main results from Tables \ref{tab:res_ncogn} and \ref{tab:res_prefs} with sharpened q-values by \citet{anderson_multiple_2008} as False Discovery Rate control for multiple hypothesis testing. The q-values are in square brackets, the cluster robust p-values are in parentheses.
		\end{tablenotes}
	\end{threeparttable}
\end{table}



%%%%%%%%%%%%%%%%%%%%%%%%%%%%%%%%%%%%%%%%%%%%%%%
\newpage
\section{Findings for additional outcomes} \label{sec:app_add_outcomes}
\setcounter{table}{0}
\setcounter{figure}{0}
\renewcommand{\thetable}{\Alph{section}.\arabic{table}}
\renewcommand\thefigure{\Alph{section}.\arabic{figure}}

\begin{table}[htbp]
	\caption{Literacy skills}
	\label{tab:res_lit_test}
	\centering
	\begin{threeparttable}
		\footnotesize
		\input{../bld/python/tables/literacy_test_scores/results_tabular_partially.tex}
		\begin{tablenotes}
			\footnotesize
			\item \textit{Notes:} Estimated average treatment effects of compulsory schooling reforms. Test scores are standardized. The sample is restricted to individuals who are older than 23 and to the following reforms: Bolivia 1994, Colombia 1991, Ghana 1961, Ghana 1987, Vietnam 1991. The \textit{linear/quadratic/cubic} model consists of reform specific \textit{linear/quadratic/cubic} cohort trends. Standard errors are cluster robust on reform $\times$ birth-year level. Estimates were obtained by plausible value method: point estimate $\theta = \frac{1}{M} \sum_{i=1}^{M}\theta_i$, where $\theta_i$ is the estimate from a regression of one plausible value $i$, and final error variance $ V = U + (1+\frac{1}{M}) B_M $, where imputation/measurement variance $B_M = \frac{1}{M-1} \sum_{i=1}^{M} (\theta - \theta_i)^2 $ and $U$ is the sampling variance (average of each plausible value's sampling variance). \textit{Data: STEP} $^{***} p < 0.01, ^{**} p < 0.05, ^{*} p < 0.1$
		\end{tablenotes}
	\end{threeparttable}
\end{table}

In addition to study the effects of compulsory education on non-cognitive skills, the STEP data set allows us to study the effect on cognitive skills. The data provides one objective measure for cognitive skills: a literacy proficiency assessment. This literacy test is assessing individuals' competence at accessing, interpreting, and evaluating information from written text. It has been designed and graded in close cooperation with the Educational Testing Service (ETS). We standardize the test scores as well for our analysis. In our analysis, we find a positive effect of the reforms on the literacy test score as can be seen in Table \ref{tab:res_lit_test}. In our preferred model (column (7)), individuals affected by the reform score 0.18 of a standard deviation better. Again, the first four treated cohorts in Vietnam do not experience this improvement.


\begin{table}[htbp]
	\caption{Wage returns to skills}
	\label{tab:wage_returns}
	\begin{center}
		\input{../bld/python/tables/analysis_returns_ln_earnings_h_usd.tex}
	\end{center}
	\begin{tablenotes}
		\footnotesize
		\item \textit{Notes:} Dependent variable: ln(wage) in USD. The sample contains individuals in Bolivia, Colombia, Ghana, and Vietnam and who are older than 23. Standard errors are heteroskedasticity robust. \textit{Data: STEP}
	\end{tablenotes}
\end{table}


\begin{table}[htbp]
	\caption{Labor market outcomes}
	\label{tab:lm_outcomes}
	\centering
	\begin{threeparttable}
		\footnotesize
		\input{../bld/python/tables/with_partially_treated/common_trend/results_All_lm_outcomes_one_table.tex}
		\begin{tablenotes}
			\footnotesize
			\item \textit{Notes:} The sample is restricted to individuals who are older than 23 and to the following reforms: Bolivia 1994, Colombia 1991, Ghana 1961, Ghana 1987, Vietnam 1991. The \textit{linear/quadratic/cubic} model consists of reform specific \textit{linear/quadratic/cubic} cohort trends. Standard errors are cluster robust on reform $\times$ birth-year level. \textit{Data: STEP} $^{***} p < 0.01, ^{**} p < 0.05, ^{*} p < 0.1$
		\end{tablenotes}
	\end{threeparttable}
\end{table}


\begin{figure}[htbp]
	\centering
	\caption{Histogram of occupations}
	\label{fig:occupations}
	\includegraphics[width = 0.95\columnwidth, trim = {0cm 2cm 0.5cm 2cm}, clip]{../bld/python/figures/occupations_isco.png}
	\caption*{\footnotesize \textit{Notes:} Occupations of individuals who worked within the past 12 months based on the International Standard Classification of Occupations 2008 (ISCO-08). \textit{Data: STEP}}
\end{figure}


\end{document}
